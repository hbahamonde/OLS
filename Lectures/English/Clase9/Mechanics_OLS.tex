%----------------------------------------------------------------------------------------
%   PACKAGES AND OTHER DOCUMENT CONFIGURATIONS
%----------------------------------------------------------------------------------------

\documentclass[10pt]{article}
\usepackage{lipsum} % Package to generate dummy text throughout this template

%\usepackage[light, math]{iwona}
%\usepackage[sc]{mathpazo} % Use the Palatino font
\usepackage[T1]{fontenc} % Use 8-bit encoding that has 256 glyphs
\linespread{1.05} % Line spacing - Palatino needs more space between lines
\usepackage{microtype} % Slightly tweak font spacing for aesthetics

\usepackage[hmarginratio=1:1,top=32mm,columnsep=20pt]{geometry} % Document margins
\usepackage{multicol} % Used for the two-column layout of the document
\usepackage[hang, small,labelfont=bf,up,textfont=it,up]{caption} % Custom captions under/above floats in tables or figures
\usepackage{booktabs} % Horizontal rules in tables
\usepackage{float} % Required for tables and figures in the multi-column environment - they need to be placed in specific locations with the [H] (e.g. \begin{table}[H])

\usepackage{lettrine} % The lettrine is the first enlarged letter at the beginning of the text
\usepackage{paralist} % Used for the compactitem environment which makes bullet points with less space between them

\usepackage{abstract} % Allows abstract customization
\renewcommand{\abstractnamefont}{\normalfont\bfseries} % Set the "Abstract" text to bold
\renewcommand{\abstracttextfont}{\normalfont\small\itshape} % Set the abstract itself to small italic text

\usepackage{titlesec} % Allows customization of titles
\renewcommand\thesection{\Roman{section}} % Roman numerals for the sections
\renewcommand\thesubsection{\Roman{subsection}} % Roman numerals for subsections
\titleformat{\section}[block]{\large\scshape\centering}{\thesection.}{1em}{} % Change the look of the section titles
\titleformat{\subsection}[block]{\large}{\thesubsection.}{1em}{} % Change the look of the section titles

\usepackage{fancybox, fancyvrb, calc}
\usepackage[svgnames]{xcolor}


%----------------------------------------------------------------------------------------
%   DOCUMENT ID (Department, Professor, Course, etc.) 
%----------------------------------------------------------------------------------------

\usepackage{fancyhdr} % Headers and footers
\pagestyle{fancy} % All pages have headers and footers
\fancyhead{} % Blank out the default header
\fancyfoot{} % Blank out the default footer
\fancyhead[C]{Mechanics of OLS} % Custom header text
\fancyfoot[RO,LE]{\thepage} % Custom footer text

%----------------------------------------------------------------------------------------
%   MY PACKAGES 
%----------------------------------------------------------------------------------------

\usepackage{amsmath}    
%\usepackage{rotating}
\usepackage{textcomp}
\usepackage{caption}
\usepackage{etex}
%\usepackage[export]{adjustbox}
%\usepackage{afterpage}
%\usepackage{filecontents}
\usepackage{color}
\usepackage{latexsym}
\usepackage{lscape}             %\begin{landscape} and \end{landscape}
\usepackage{amsfonts}
%\usepackage{mathabx}
\usepackage{amssymb}
%\usepackage{dashrule}
%\usepackage{txfonts}
%\usepackage{pgfkeys}
%\usepackage{framed}
\usepackage{tree-dvips}
\usepackage{caption}
%\usepackage{fancyvrb}
%\usepackage{pgffor}
\usepackage{xcolor}
%\usepackage{pxfonts}
\usepackage{wasysym}
\usepackage{authblk}
%\usepackage{paracol}
\usepackage{setspace}
%\usepackage{qtree}
%\usepackage{tree-dvips}
\usepackage{sgame}              % shouldn't have neither array nor tabularx packages
\usepackage{tikz}
%\usetikzlibrary{trees}
\usepackage[latin1]{inputenc}
%\label{tab:1}      %\autoref{tab:1}    %ocupar para citar.
% \hyperlik{table1} \hypertarget{table1} 
% \textquoteright           %apostrofe
\usepackage{hyperref}       %desactivar para link rojos
\usepackage{natbib}
%\usepackage{proof}             %for proofs





%----------------------------------------------------------------------------------------
%   Other ADDS-ON
%----------------------------------------------------------------------------------------

% independence symbol \independent
\newcommand\independent{\protect\mathpalette{\protect\independenT}{\perp}}
\def\independenT#1#2{\mathrel{\rlap{$#1#2$}\mkern2mu{#1#2}}}


% VERBATIM WITH BACKGROUND COLOR
\newenvironment{colframe}{%
  \begin{Sbox}
    \begin{minipage}
      {\columnwidth%-\leftmargin-\rightmargin-6pt
      }
    }{%
    \end{minipage}
  \end{Sbox}
  \begin{center}
    \colorbox{LightSteelBlue}{\TheSbox}
  \end{center}
}


\hypersetup{
    bookmarks=true,         % show bookmarks bar?
    unicode=false,          % non-Latin characters in Acrobat$'$s bookmarks
    pdftoolbar=true,        % show Acrobat$'$s toolbar?
    pdfmenubar=true,        % show Acrobat$'$s menu?
    pdffitwindow=false,     % window fit to page when opened
    pdfstartview={FitH},    % fits the width of the page to the window
    pdftitle={My title},    % title
    pdfauthor={Author},     % author
    pdfsubject={Subject},   % subject of the document
    pdfcreator={Creator},   % creator of the document
    pdfproducer={Producer}, % producer of the document
    pdfkeywords={keyword1} {key2} {key3}, % list of keywords
    pdfnewwindow=true,      % links in new window
    colorlinks=true,       % false: boxed links; true: colored links
    linkcolor=ForestGreen,          % color of internal links (change box color with linkbordercolor)
    citecolor=ForestGreen,        % color of links to bibliography
    filecolor=ForestGreen,      % color of file links
    urlcolor=ForestGreen           % color of external links
}


% PROPOSITIONS
\newtheorem{proposition}{Proposition}

%\linespread{1.5}

%----------------------------------------------------------------------------------------
%   TITLE SECTION
%----------------------------------------------------------------------------------------

%\title{\vspace{-15mm}\fontsize{18pt}{7pt}\selectfont\textbf{Experimental Economists and Psychologists: Two Worlds Apart}} % Article title

%\author[1]{
%\large
%\textsc{H\'ector Bahamonde}\\ 
%\thanks{}
%\normalsize Political Science Dpt. $\bullet$ Rutgers University \\ % Your institution
%\normalsize \texttt{e:}\href{mailto:hector.bahamonde@rutgers.edu}{\texttt{hector.bahamonde@rutgers.edu}}\\
%\normalsize \texttt{w:}\href{http://www.hectorbahamonde.com}{\texttt{www.hectorbahamonde.com}}
%\vspace{-5mm}
%}
%\date{\today}

%----------------------------------------------------------------------------------------

\begin{document}

%\maketitle % Insert title


\thispagestyle{fancy} % All pages have headers and footers

%----------------------------------------------------------------------------------------
%   ABSTRACT
%----------------------------------------------------------------------------------------

%\begin{abstract}
%   ABSTRACT
%\end{abstract}


%----------------------------------------------------------------------------------------
%   CONTENT
%----------------------------------------------------------------------------------------

%\graphicspath{
%{/Users/hectorbahamonde/RU/Term5/Experiments_Redlawsk/Experiment/Data/}
%}
\hspace{-5mm}{\bf Professor}: H\'ector Bahamonde.\\
\texttt{e:}\href{mailto:hibano@utu.fi}{\texttt{hibano@utu.fi}}\\
\texttt{w:}\href{http://www.hectorbahamonde.com}{\texttt{www.hectorbahamonde.com}}\\
{\bf Course}: OLS.


\paragraph{The ``Mechanic'' behind OLS} Let's think about the relationship \emph{schooling} and \emph{earnings}, \emph{controlling} for \emph{experience}. What does it mean ``to control for'' something in this context? 
\\
\\
Now, let's suppose we have the following data,

\begin{table}[!h]
        \centering
        
\begin{tabular}{|p{0.16\textwidth}|p{0.20\textwidth}|p{0.25\textwidth}|p{0.39\textwidth}|}
\hline 
 \begin{center}
\textbf{Name (i)}
\end{center}
 & \begin{center}
\textbf{Earnings (Y)}
\end{center}
 & \begin{center}
\textbf{Education (x1)}
\end{center}
 & \begin{center}
\textbf{Experience (x2)}
\end{center}
 \\
\hline 
 \begin{center}
Alfred
\end{center}
 & \begin{center}
3
\end{center}
 & \begin{center}
2
\end{center}
 & \begin{center}
2
\end{center}
 \\
\hline 
 \begin{center}
Brandon
\end{center}
 & \begin{center}
5
\end{center}
 & \begin{center}
7
\end{center}
 & \begin{center}
4
\end{center}
 \\
\hline 
 \begin{center}
Charly
\end{center}
 & \begin{center}
7
\end{center}
 & \begin{center}
3
\end{center}
 & \begin{center}
6
\end{center}
 \\
 \hline
\end{tabular}
        
        \end{table}


\paragraph{Hypothesis:} ``The more education, the higher earnings,'' for the  average level of experience (And what do I mean by ``for the  average level of experience'' and \emph{Why does it matter?}) 



\section{By how much do my earnings rise if my schooling goes up?}

\paragraph{}The linear model is given by the next formula,



\begin{gather*}
 \text{Earnings}_{i} = \beta 0+\beta_{1}\text{Education}_{i} \ +\ \beta_{2}\text{Experience}_{i} + e_{i}
\end{gather*}





\begin{itemize}
\item \textbf{What we observe:} $\displaystyle {\boldsymbol x}$ and $\displaystyle y$. 
\item \textbf{What we don't observe, but should estimate}: \ $\displaystyle \boldsymbol{\beta} $ and $\displaystyle \epsilon $.
\end{itemize}



\paragraph{}{\Large \textbf{Let's revisit the formula, but this time in matrix form:}}



\paragraph{}Let's define what we know:


\begin{gather*}
Y\ =\begin{bmatrix}
3\\
5\\
7
\end{bmatrix}\\
\\
\\
\ X=\begin{bmatrix}
2 & 2\\
7 & 4\\
3 & 6
\end{bmatrix}
\end{gather*}
\begin{equation*}
\end{equation*}


\paragraph{}...and see how OLS looks like but in matrix form:


\begin{equation*}
\begin{bmatrix}
3\\
5\\
7
\end{bmatrix}_{y} \ =\ \beta 0+\beta 1\begin{bmatrix}
2\\
7\\
3
\end{bmatrix}_{x1} \ +\ \beta 2\begin{bmatrix}
2\\
4\\
6
\end{bmatrix}_{x2} +e_{i}
\end{equation*}

It's easy to see that:

\begin{itemize}
\item We should multiply $\displaystyle \beta 1$ times $\displaystyle x$1 and $\displaystyle \beta 2$ times $\displaystyle x$2.

\item $\displaystyle \beta $0, $\displaystyle \beta $1, $\displaystyle \beta $2 and $\displaystyle \epsilon_{i}$ unknown quantities. Hence, we should infer/estimate that (this is \emph{inferential} statistics!). 

\item $\displaystyle \beta $0, $\displaystyle \beta $1 and $\displaystyle \beta $2 are scalars (single numbers and constants), where the vector containing all estimations is defined as, $\boldsymbol{\beta} \ =\ \begin{bmatrix} \beta_{0}\\\beta_{1}\\\beta_{2}\end{bmatrix}_{\boldsymbol{\beta}}$

\item The only parameter that's indexed (i.e. one row per observation or ``individual,'' hence the tiny ``\emph{i}'') is $\displaystyle \epsilon _{i}$. We will address this in another class. But basically, it's the ``error'' or ``residual.'' 
\\
\\
Let's think about the case of ``Aldred.''  If,
\\
\\
$\beta 0\ =\ -3$, 
\\
\\
$\beta 1\ =\ 1$, and
\\
\\
$\beta 2\ =\ 2$, then,
\\
\\
we have that,
\\
\\
$\displaystyle \hat{y}_{_{\text{Alfred}}} -3\ +\ 1( 2) \ +\ $2(2) = 3. Then if $y=3$ and $\hat{y}=3$, {\bf by how much did I miss my prediction?} Thus:
\end{itemize}


\begin{equation*}
\hat{y}_{\text{Alfred}} \ =\ 3\ =\ -3\ +1( 2) \ +2( 2) \ +\ 0
\end{equation*}


\paragraph{}Here, $\displaystyle \epsilon _{\text{Alfred}}$=0. In this sense, $\displaystyle e_{i}$ is just the difference between what we observe and what estimate in the model. ``Philosophically'' it means more than that, but we'll talk about this soon.



\paragraph{}Let's continue...





\paragraph{}{\Large \textbf{Deriving} \;$\displaystyle \beta $\textbf{ (quantitative effect of} $\displaystyle X$ \textbf{on} $\displaystyle Y$\textbf{):}}
\\
\\
$\displaystyle \beta \ =\ \left( \boldsymbol{X}^{T} \boldsymbol{X}\right)^{-1} \boldsymbol{X}^{T} y$



\paragraph{}Let's do this by hand...:


\begin{gather*}
y\ =\begin{bmatrix}
3\\
5\\
7
\end{bmatrix}\\
\\
\\
\ \boldsymbol{X}=\begin{bmatrix}
1 & 2 & 2\\
1 & 7 & 4\\
1 & 3 & 6
\end{bmatrix}\\
\\
\\
\boldsymbol{X}^{T} =\begin{bmatrix}
1 & 1 & 1\\
2 & 7 & 3\\
2 & 4 & 6
\end{bmatrix}\\
\\
\boldsymbol{X}^{T} \times \boldsymbol{X}\ =\ \begin{bmatrix}
1 & 1 & 1\\
2 & 7 & 3\\
2 & 4 & 6
\end{bmatrix}_{\boldsymbol{X}^{T}} \ \times \begin{bmatrix}
1 & 2 & 2\\
1 & 7 & 4\\
1 & 3 & 6
\end{bmatrix}_{\boldsymbol{X}} \ =\ \begin{bmatrix}
3 & 12 & 12\\
12 & 62 & 50\\
12 & 50 & 56
\end{bmatrix}\\
\\
\\
\\
\left( \boldsymbol{X}^{T} \times \boldsymbol{X}\right)^{-1} =\frac{1}{\left( \boldsymbol{X}^{T} \boldsymbol{X}\right)} \ =\ \frac{1}{\text{det}\left( \boldsymbol{X}^{T} \boldsymbol{X}\right)} \ \times \ \text{Adj}\left( \boldsymbol{X}^{T} \boldsymbol{X}\right) \ =\ \begin{bmatrix}
3 & -0.22 & -0.44\\
-0.22 & 0.074 & -0.0185\\
-0.44 & -0.0185 & 0.129
\end{bmatrix}\\
\\
\\
\boldsymbol{\beta} \ =\ \left( \boldsymbol{X}^{T} \boldsymbol{X}\right)^{-1} \boldsymbol{X}^{T} y\ =\ \begin{bmatrix}
3 & -0.22 & -0.44\\
-0.22 & 0.074 & -0.0185\\
-0.44 & -0.0185 & 0.129
\end{bmatrix} \ \times \ \begin{bmatrix}
1 & 1 & 1\\
2 & 7 & 3\\
2 & 4 & 6
\end{bmatrix} \ \times \ \begin{bmatrix}
3\\
5\\
7
\end{bmatrix} \ \\
\\
\boldsymbol{\beta} =\ \begin{bmatrix}
1\\
0\\
1
\end{bmatrix}_{\boldsymbol{\beta}}\\
\end{gather*}

This means that $\displaystyle \beta 0\ =\ 1,\ \beta 1\ =\ 0$ and that $\displaystyle \beta 2\ =\ 1$. Let's re-write our formula:


\begin{gather*}
\boldsymbol{\beta} \ =\ \left( \boldsymbol{X}^{T} \boldsymbol{X}\right)^{-1} \boldsymbol{X}^{T} y\ =\ \begin{bmatrix}
1\\
0\\
1
\end{bmatrix}_{\boldsymbol{\beta}}\\
\ \text{The formula we had before: }\begin{bmatrix}
3\\
5\\
7
\end{bmatrix}_{y} \ =\ \beta 0+\beta 1\begin{bmatrix}
2\\
7\\
3
\end{bmatrix}_{x1} \ +\ \beta 2\begin{bmatrix}
2\\
4\\
6
\end{bmatrix}_{x2} +e_{i}\\
\\
\text{Results we have now:} \ \begin{bmatrix}
3\\
5\\
7
\end{bmatrix}_{y} \ =\ 1+0\begin{bmatrix}
2\\
7\\
3
\end{bmatrix}_{x1} \ +\ 1\begin{bmatrix}
2\\
4\\
6
\end{bmatrix}_{x2} +e_{i}\\
\end{gather*}

Ok, let's now turn to \texttt{R}.
\end{document}