%----------------------------------------------------------------------------------------
%   PACKAGES AND OTHER DOCUMENT CONFIGURATIONS
%----------------------------------------------------------------------------------------

\documentclass[10pt]{article}
\usepackage{lipsum} % Package to generate dummy text throughout this template

%\usepackage[light, math]{iwona}
%\usepackage[sc]{mathpazo} % Use the Palatino font
\usepackage[T1]{fontenc} % Use 8-bit encoding that has 256 glyphs
\linespread{1.05} % Line spacing - Palatino needs more space between lines
\usepackage{microtype} % Slightly tweak font spacing for aesthetics

\usepackage[hmarginratio=1:1,top=32mm,columnsep=20pt]{geometry} % Document margins
\usepackage{multicol} % Used for the two-column layout of the document
\usepackage[hang, small,labelfont=bf,up,textfont=it,up]{caption} % Custom captions under/above floats in tables or figures
\usepackage{booktabs} % Horizontal rules in tables
\usepackage{float} % Required for tables and figures in the multi-column environment - they need to be placed in specific locations with the [H] (e.g. \begin{table}[H])

\usepackage{lettrine} % The lettrine is the first enlarged letter at the beginning of the text
\usepackage{paralist} % Used for the compactitem environment which makes bullet points with less space between them

\usepackage{abstract} % Allows abstract customization
\renewcommand{\abstractnamefont}{\normalfont\bfseries} % Set the "Abstract" text to bold
\renewcommand{\abstracttextfont}{\normalfont\small\itshape} % Set the abstract itself to small italic text

\usepackage{titlesec} % Allows customization of titles
\renewcommand\thesection{\Roman{section}} % Roman numerals for the sections
\renewcommand\thesubsection{\Roman{subsection}} % Roman numerals for subsections
\titleformat{\section}[block]{\large\scshape\centering}{\thesection.}{1em}{} % Change the look of the section titles
\titleformat{\subsection}[block]{\large}{\thesubsection.}{1em}{} % Change the look of the section titles

\usepackage{fancybox, fancyvrb, calc}
\usepackage[svgnames]{xcolor}


%----------------------------------------------------------------------------------------
%   DOCUMENT ID (Department, Professor, Course, etc.) 
%----------------------------------------------------------------------------------------

\usepackage{fancyhdr} % Headers and footers
\pagestyle{fancy} % All pages have headers and footers
\fancyhead{} % Blank out the default header
\fancyfoot{} % Blank out the default footer
\fancyhead[C]{Residuos} % Custom header text
\fancyfoot[RO,LE]{\thepage} % Custom footer text

%----------------------------------------------------------------------------------------
%   MY PACKAGES 
%----------------------------------------------------------------------------------------

\usepackage{amsmath}    
%\usepackage{rotating}
\usepackage{textcomp}
\usepackage{caption}
\usepackage{etex}
%\usepackage[export]{adjustbox}
%\usepackage{afterpage}
%\usepackage{filecontents}
\usepackage{color}
\usepackage{latexsym}
\usepackage{lscape}             %\begin{landscape} and \end{landscape}
\usepackage{amsfonts}
%\usepackage{mathabx}
\usepackage{amssymb}
%\usepackage{dashrule}
%\usepackage{txfonts}
%\usepackage{pgfkeys}
%\usepackage{framed}
\usepackage{tree-dvips}
\usepackage{caption}
%\usepackage{fancyvrb}
%\usepackage{pgffor}
\usepackage{xcolor}
%\usepackage{pxfonts}
\usepackage{wasysym}
\usepackage{authblk}
%\usepackage{paracol}
\usepackage{setspace}
%\usepackage{qtree}
%\usepackage{tree-dvips}
\usepackage{sgame}              % shouldn't have neither array nor tabularx packages
\usepackage{tikz}
%\usetikzlibrary{trees}
\usepackage[latin1]{inputenc}
%\label{tab:1}      %\autoref{tab:1}    %ocupar para citar.
% \hyperlik{table1} \hypertarget{table1} 
% \textquoteright           %apostrofe
\usepackage{hyperref}       %desactivar para link rojos
\usepackage{natbib}
%\usepackage{proof}             %for proofs





%----------------------------------------------------------------------------------------
%   Other ADDS-ON
%----------------------------------------------------------------------------------------

% independence symbol \independent
\newcommand\independent{\protect\mathpalette{\protect\independenT}{\perp}}
\def\independenT#1#2{\mathrel{\rlap{$#1#2$}\mkern2mu{#1#2}}}


% VERBATIM WITH BACKGROUND COLOR
\newenvironment{colframe}{%
  \begin{Sbox}
    \begin{minipage}
      {\columnwidth%-\leftmargin-\rightmargin-6pt
      }
    }{%
    \end{minipage}
  \end{Sbox}
  \begin{center}
    \colorbox{LightSteelBlue}{\TheSbox}
  \end{center}
}


\hypersetup{
    bookmarks=true,         % show bookmarks bar?
    unicode=false,          % non-Latin characters in Acrobat$'$s bookmarks
    pdftoolbar=true,        % show Acrobat$'$s toolbar?
    pdfmenubar=true,        % show Acrobat$'$s menu?
    pdffitwindow=false,     % window fit to page when opened
    pdfstartview={FitH},    % fits the width of the page to the window
    pdftitle={My title},    % title
    pdfauthor={Author},     % author
    pdfsubject={Subject},   % subject of the document
    pdfcreator={Creator},   % creator of the document
    pdfproducer={Producer}, % producer of the document
    pdfkeywords={keyword1} {key2} {key3}, % list of keywords
    pdfnewwindow=true,      % links in new window
    colorlinks=true,       % false: boxed links; true: colored links
    linkcolor=ForestGreen,          % color of internal links (change box color with linkbordercolor)
    citecolor=ForestGreen,        % color of links to bibliography
    filecolor=ForestGreen,      % color of file links
    urlcolor=ForestGreen           % color of external links
}


% PROPOSITIONS
\newtheorem{proposition}{Proposition}

%\linespread{1.5}

%----------------------------------------------------------------------------------------
%   TITLE SECTION
%----------------------------------------------------------------------------------------

%\title{\vspace{-15mm}\fontsize{18pt}{7pt}\selectfont\textbf{Experimental Economists and Psychologists: Two Worlds Apart}} % Article title

%\author[1]{
%\large
%\textsc{H\'ector Bahamonde}\\ 
%\thanks{}
%\normalsize Political Science Dpt. $\bullet$ Rutgers University \\ % Your institution
%\normalsize \texttt{e:}\href{mailto:hector.bahamonde@rutgers.edu}{\texttt{hector.bahamonde@rutgers.edu}}\\
%\normalsize \texttt{w:}\href{http://www.hectorbahamonde.com}{\texttt{www.hectorbahamonde.com}}
%\vspace{-5mm}
%}
%\date{\today}

%----------------------------------------------------------------------------------------

\begin{document}

%\maketitle % Insert title


\thispagestyle{fancy} % All pages have headers and footers

%----------------------------------------------------------------------------------------
%   ABSTRACT
%----------------------------------------------------------------------------------------

%\begin{abstract}
%   ABSTRACT
%\end{abstract}


%----------------------------------------------------------------------------------------
%   CONTENT
%----------------------------------------------------------------------------------------

%\graphicspath{
%{/Users/hectorbahamonde/RU/Term5/Experiments_Redlawsk/Experiment/Data/}
%}
\hspace{-5mm}{\bf Profesor}: H\'ector Bahamonde.\\
\texttt{e:}\href{mailto:hector.bahamonde@uoh.cl}{\texttt{hector.bahamonde@uoh.cl}}\\
\texttt{w:}\href{http://www.hectorbahamonde.com}{\texttt{www.hectorbahamonde.com}}\\
{\bf Curso}: OLS.


\paragraph{La Mec\'anica detras del OLS} Pensemos en la relaci\'on entre \emph{educaci\'on} e \emph{ingreso}, ``controlando por'' \emph{a\~nos de experiencia laboral}. Supongamos que tenemos la siguiente base de datos:


\begin{table}[!h]
        \centering
        
\begin{tabular}{|p{0.16\textwidth}|p{0.20\textwidth}|p{0.25\textwidth}|p{0.39\textwidth}|}
\hline 
 \begin{center}
\textbf{Nombre (i)}
\end{center}
 & \begin{center}
\textbf{Ingreso (Y)}
\end{center}
 & \begin{center}
\textbf{Educacion (X1)}
\end{center}
 & \begin{center}
\textbf{Experiencia (X2)}
\end{center}
 \\
\hline 
 \begin{center}
Pedro
\end{center}
 & \begin{center}
3
\end{center}
 & \begin{center}
2
\end{center}
 & \begin{center}
2
\end{center}
 \\
\hline 
 \begin{center}
Juan
\end{center}
 & \begin{center}
5
\end{center}
 & \begin{center}
7
\end{center}
 & \begin{center}
4
\end{center}
 \\
\hline 
 \begin{center}
Diego
\end{center}
 & \begin{center}
7
\end{center}
 & \begin{center}
3
\end{center}
 & \begin{center}
6
\end{center}
 \\
 \hline
\end{tabular}
        
        \end{table}


\paragraph{Hypotesis:} ``A m\'as educaci\'on, m\'as ingreso, para el promedio de a\~nos de experiencia laboral''. {\color{red}A que se refiere ``para el promedio de a\~nos promedio''?}


\section{Cu\'anto sube mi ingreso si aumento mi educacion? }

\paragraph{}El modelo de regresi\'on lineal esta dado por la siguiente formula:


\begin{center}

\begin{gather}
Y_{i} \ =\ \beta 0+\beta 1X1_{i} \ +\ \beta 2X2_{i} +e_{i} \notag\\
 \notag
\end{gather}
\end{center}




\begin{itemize}
\item \textbf{Lo que conocemos:} $\displaystyle X$ e $\displaystyle Y$. 
\item \textbf{Lo que no conocemos, pero debemos estimar}: \ $\displaystyle \beta $ y $\displaystyle \epsilon $.
\end{itemize}



\paragraph{}{\Large \textbf{Volvamos a repensar la formula del modelo de regresi\'on, pero en t\'erminos de matrices:}}



\paragraph{}Definamos lo que conocemos:


\begin{gather*}
Y\ =\begin{bmatrix}
3\\
5\\
7
\end{bmatrix}\\
\\
\\
\ X=\begin{bmatrix}
2 & 2\\
7 & 4\\
3 & 6
\end{bmatrix}
\end{gather*}
\begin{equation*}
\end{equation*}


\paragraph{}...y veamos c\'omo se ve OLS pero con matrices:


\begin{equation*}
\begin{bmatrix}
3\\
5\\
7
\end{bmatrix} \ =\ \beta 0+\beta 1\begin{bmatrix}
2\\
7\\
3
\end{bmatrix} \ +\ \beta 2\begin{bmatrix}
2\\
4\\
6
\end{bmatrix} +e_{i}
\end{equation*}


\begin{itemize}
\item Es f\'acil ver que:
\item Debemos multiplicar $\displaystyle \beta 1$ por $\displaystyle X$1 y $\displaystyle \beta 2$ $\displaystyle X$2.
\item $\displaystyle \beta $0, $\displaystyle \beta $1, $\displaystyle \beta $2 y $\displaystyle \epsilon $ son cantidades desconocidas, pero que debemos estimar.
\item $\displaystyle \beta $0, $\displaystyle \beta $1y $\displaystyle \beta $2 son constantes.
\item El \'unico par\'ametro que est\'a indexado, es $\displaystyle \epsilon _{i}$. Otra clase hablaremos de esto. \textbf{Adelanto}: Pensemos el caso de ``Pedro''. Si por ejemplo $\displaystyle \beta 0\ =\ -3$, $\displaystyle \beta 1\ =\ 1$ y $\displaystyle \beta 2\ =\ 2$, tendremos que $\displaystyle y_{_{\text{Pedro}}} -3\ +\ 1( 2) \ +\ $2(2) = 3. Entonces:
\end{itemize}


\begin{equation*}
Y_{\text{Pedro}} \ =\ 3\ =\ -3\ +1( 2) \ +2( 2) \ +\ 0
\end{equation*}


\paragraph{}Aqu\'i $\displaystyle \epsilon _{\text{Pedro}}$=0. En ese sentido, $\displaystyle e_{i}$ es la diferencia entre lo que estimamos y lo que observamos. ``Filosoficamente'', significa otra cosa.



\paragraph{}Continuemos.





\paragraph{}{\Large \textbf{Formula para sacar }$\displaystyle \beta $\textbf{ (el efecto de }$\displaystyle X$\textbf{ sobre }$\displaystyle Y$\textbf{):}}



$\displaystyle \beta \ =\ \left( X^{T} X\right)^{-1} X^{T} Y$



\paragraph{}Recapitulemos lo observado, y hagamos el c\'alculo:


\begin{gather*}
Y\ =\begin{bmatrix}
3\\
5\\
7
\end{bmatrix}\\
\\
\\
\ X=\begin{bmatrix}
1 & 2 & 2\\
1 & 7 & 4\\
1 & 3 & 6
\end{bmatrix}\\
\\
\\
X^{T} =\begin{bmatrix}
1 & 1 & 1\\
2 & 7 & 3\\
2 & 4 & 6
\end{bmatrix}\\
\\
X^{T} \times X\ =\ \begin{bmatrix}
1 & 1 & 1\\
2 & 7 & 3\\
2 & 4 & 6
\end{bmatrix} \ \times \begin{bmatrix}
1 & 2 & 2\\
1 & 7 & 4\\
1 & 3 & 6
\end{bmatrix} \ =\ \begin{bmatrix}
3 & 12 & 12\\
12 & 62 & 50\\
12 & 50 & 56
\end{bmatrix}\\
\\
\\
\\
\left( X^{T} \times X\right)^{-1} =\frac{1}{\left( X^{T} X\right)} \ =\ \frac{1}{\text{det}\left( X^{T} X\right)} \ \times \ \text{Adj}\left( X^{T} X\right) \ =\ \begin{bmatrix}
3 & -0.22 & -0.44\\
-0.22 & 0.074 & -0.0185\\
-0.44 & -0.0185 & 0.129
\end{bmatrix}\\
\\
\\
\beta \ =\ \left( X^{T} X\right)^{-1} X^{T} Y\ =\ \begin{bmatrix}
3 & -0.22 & -0.44\\
-0.22 & 0.074 & -0.0185\\
-0.44 & -0.0185 & 0.129
\end{bmatrix} \ \times \ \begin{bmatrix}
1 & 1 & 1\\
2 & 7 & 3\\
2 & 4 & 6
\end{bmatrix} \ \times \ \begin{bmatrix}
3\\
5\\
7
\end{bmatrix} \ \\
\\
\beta =\ \begin{bmatrix}
1\\
0\\
1
\end{bmatrix}\\
\end{gather*}
Esto quiere decir que $\displaystyle \beta 0\ =\ 1,\ \beta 1\ =\ 0$ y $\displaystyle \beta 2\ =\ 1$. Volvamos a re-escribir nuestra formula:




\begin{gather*}
\beta \ =\ \left( X^{T} X\right)^{-1} X^{T} Y\ =\ \begin{bmatrix}
1\\
0\\
1
\end{bmatrix}\\
\ \text{La formula que teniamos antes: }\begin{bmatrix}
3\\
5\\
7
\end{bmatrix} \ =\ \beta 0+\beta 1\begin{bmatrix}
2\\
7\\
3
\end{bmatrix} \ +\ \beta 2\begin{bmatrix}
2\\
4\\
6
\end{bmatrix} +e_{i}\\
\\
\text{Los resultados que tenemos ahora:} \ \begin{bmatrix}
3\\
5\\
7
\end{bmatrix} \ =\ 1+0\begin{bmatrix}
2\\
7\\
3
\end{bmatrix} \ +\ 1\begin{bmatrix}
2\\
4\\
6
\end{bmatrix} +e_{i}\\
\end{gather*}
\end{document}