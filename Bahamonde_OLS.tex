%LaTeX Curriculum Vitae Template
%
% Copyright (C) 2004-2009 Jason Blevins <jrblevin@sdf.lonestar.org>
% http://jblevins.org/projects/cv-template/
%
% You may use use this document as a template to create your own CV
% and you may redistribute the source code freely. No attribution is
% required in any resulting documents. I do ask that you please leave
% this notice and the above URL in the source code if you choose to
% redistribute this file.

\documentclass[letterpaper]{article}

\usepackage{hyperref}
\hypersetup{
    bookmarks=true,         % show bookmarks bar?
    unicode=false,          % non-Latin characters in Acrobat’s bookmarks
    pdftoolbar=true,        % show Acrobat’s toolbar?
    pdfmenubar=true,        % show Acrobat’s menu?
    pdffitwindow=true,     % window fit to page when opened
    pdfstartview={FitH},    % fits the width of the page to the window
    pdftitle={My title},    % title
    pdfauthor={Author},     % author
    pdfsubject={Subject},   % subject of the document
    pdfcreator={Creator},   % creator of the document
    pdfproducer={Producer}, % producer of the document
    pdfkeywords={keyword1} {key2} {key3}, % list of keywords
    pdfnewwindow=true,      % links in new window
    colorlinks=true,       % false: boxed links; true: colored links
    linkcolor=blue,          % color of internal links (change box color with linkbordercolor)
    citecolor=blue,        % color of links to bibliography
    filecolor=blue,      % color of file links
    urlcolor=blue           % color of external links
}



\usepackage{geometry}
\usepackage{import} % To import email.
\usepackage{marvosym} % face package
%\usepackage{xcolor,color}
\usepackage{fontawesome}
\usepackage{amssymb} % for bigstar
\usepackage{epigraph}
\usepackage{float}
\usepackage[svgnames]{xcolor}

% Comment the following lines to use the default Computer Modern font
% instead of the Palatino font provided by the mathpazo package.
% Remove the 'osf' bit if you don't like the old style figures.
\usepackage[T1]{fontenc}
\usepackage[sc,osf]{mathpazo}

% Set your name here
\def\name{Quantitative Methods (VALT6307)}

% Replace this with a link to your CV if you like, or set it empty
% (as in \def\footerlink{}) to remove the link in the footer:
\def\footerlink{}
% \href{http://www.hectorbahamonde.com}{www.HectorBahamonde.com}

% The following metadata will show up in the PDF properties
\hypersetup{
  colorlinks = true,
  urlcolor = blue,
  pdfauthor = {\name},
  pdfkeywords = {intro to social sciences},
  pdftitle = {\name: Syllabus},
  pdfsubject = {Syllabus},
  pdfpagemode = UseNone
}

\geometry{
  body={6.5in, 8.5in},
  left=1.0in,
  top=1.25in
}

% Customize page headers
\pagestyle{myheadings}
\markright{{\tiny \name}}
\thispagestyle{empty}

% Custom section fonts
\usepackage{sectsty}
\sectionfont{\rmfamily\mdseries\Large}
\subsectionfont{\rmfamily\mdseries\itshape\large}

% Don't indent paragraphs.
\setlength\parindent{0em}

% Make lists without bullets
\renewenvironment{itemize}{
  \begin{list}{}{
    \setlength{\leftmargin}{1.5em}
  }
}{
  \end{list}
}


%%% bib begin
\usepackage[american]{babel}
\usepackage{csquotes}
%\usepackage[style=chicago-authordate,doi=false,isbn=false,url=false,eprint=false]{biblatex}

\usepackage[authordate,isbn=false,doi=false,url=false,eprint=false]{biblatex-chicago}
\DeclareFieldFormat[article]{title}{\mkbibquote{#1}} % make article titles in quotes
\DeclareFieldFormat[thesis]{title}{\mkbibemph{#1}} % make theses italics

\AtEveryBibitem{\clearfield{month}}
\AtEveryCitekey{\clearfield{month}}

\addbibresource{/Users/hectorbahamonde/Bibliografia_PoliSci/library.bib} 
\addbibresource{/Users/hectorbahamonde/Bibliografia_PoliSci/Bahamonde_BibTex2013.bib} 

% USAGES
%% use \textcite to cite normal
%% \parencite to cite in parentheses
%% \footcite to cite in footnote
%% the default can be modified in autocite=FOO, footnote, for ex. 
%%% bib end




\begin{document}

% Place name at left
%{\huge \name}

% Alternatively, print name centered and bold:
\centerline{\huge \bf \name}

\epigraph{\emph{Statistics: ``science dealing with data about the condition of a state or community''}}{Gottfried Aschenwall, 1770}


\vspace{0.25in}

\begin{minipage}{0.45\linewidth}
 University of Turku \\
  Faculty of Social Sciences \\
  Turku, Finland\\
  \\
  \\

\end{minipage}
\hspace{4cm}\begin{minipage}{0.45\linewidth}
  \begin{tabular}{ll}
{\bf Last updated}: \today. \\
 {\bf Download last version} \href{https://github.com/hbahamonde/OLS/raw/master/Bahamonde_OLS.pdf}{here}.%\\
   %{\bf {\color{red}{\scriptsize Not intended as a definitive version}}} %\\
    \\
    \\
    \\
    \\
    \\
  \end{tabular}
\end{minipage}

\subsection*{General Overview}


\vspace{1mm}
{\bf Professor}: H\'ector Bahamonde, PhD.\\
\texttt{e:}\href{mailto:hibano@utu.fi}{\texttt{hibano@utu.fi}}\\
\texttt{w:}\href{http://www.hectorbahamonde.com}{\texttt{www.HectorBahamonde.com}}\\
{\bf Office Hours}: Schedule time with me \href{https://calendly.com/bahamonde}{\texttt{here}}.\\

\vspace{5mm}
{\bf Place}: Computer Lab \texttt{PUB-409}.\\
{\bf Time}: Lectures will take place during the following March dates: 10th., 17th., 24th. and 31st. All of them from 10.30 AM to 1.30 PM.\\

{\bf Course website}: \href{https://moodle.utu.fi/course/view.php?id=24005}{Moodle}.

\vspace{5mm}
{\bf TA}: Valtteri Pulkkinen.\\
\texttt{e}: \href{mailto:valtteri.s.pulkkinen@utu.fi}{\texttt{valtteri.s.pulkkinen@utu.fi}}\\
{\bf TA Bio}: He is a soon-to-be MA in Political Science and the TA for this course. Pulkkinen is especially interested in quantitative methods, private-public cooperation and public affairs. You can email him for help before and during this course.\\



\vspace{5mm}
{\bf Program}:  Master of Social Sciences, University of Turku.\\
{\bf Semester}: Spring.\\
{\bf Credits}: 2.





\subsection*{Motivation: Why take this course?}

\emph{What's the effect of education on income? How can we evaluate the effectiveness of a public policy? Does legalizing some drug increase its consumption? Which political candidate will win the election?} All these questions entail some kind of relationship between two social phenomenon (a.k.a. \emph{correlation}). In this course we will learn (1) how to answer these questions from a quantitative perspective, (2) how to select the ``right'' quantitative method depending on the type of data we have (a.k.a. \emph{functional form}), (3) how to quantify the amount of \emph{uncertainty} we have (and why it matters), and (4) how to communicate quantitative results effectively (even for non-specialized audiences). 
\\
\\
Public entities guide their strategic decisions based on quantifiable information, i.e., data. This decision-making process has taken even much more relevance nowadays where there has been a wave of data digitization, making available much more high-quality data. I believe this is a great opportunity for social scientists like ourselves, putting heavy pressures on us to learn how to analyze those data. If ten years ago we complained that there was not enough data, our problem now is different: there are so much data that we need to learn how to analyze it.
\\
\\
Though what we will learn this semester is highly mathematical and numeric, and thus, it might seem to you as ``very scientific'' or ``irrefutable,'' don't get confused, these methods are \emph{not} ``bullet-proof:'' they will \emph{never} ``proof'' anything at all. During the semester, we will learn \emph{inferential} statistics, that means that everything we do will come with some degree of \emph{\bf uncertainty}. All the time, we will also rely on \emph{untestable} \emph{\bf assumptions}. {\bf As we shall see, inferential statistics is more art than science}.
\\
\\
Depending on our progress during the semester, we will pay special attention to an issue that is absolutely relevant nowadays in applied social sciences: \emph{causal inference}. Experiments are the gold-standard for making causal claims. However, often times conducting experiments is either too expensive, unethical, or impossible. Under the ``right'' circumstances, though, some times it is possible to get quasi-experimental statistical designs that might get us closer to the gold-standard. We will also discuss why the methods we will learn this semester are \emph{not} causal, i.e., \emph{correlation is not causation}.
\\
\\
Honestly, I hope this course captivates your enthusiasm, and gets you interested and curious about ways to study different social phenomena from a quantitative perspective, \emph{tervetuloa}!


\subsection*{Description}

Enrolled students will acquire a basic inferential statistics toolkit. The course will pay special attention to Ordinary Least Square regression (OLS) and a selection of Generalised Linear Models (such as logit/probit, multinomial, ordered models and/or rare events data generating processes)---the workhorses of quantitative social sciences. This course is very hands-on, and while some statistical theory will be covered, the core of it will be on data analyses and programming in \texttt{R}. 
\\
\\
Overall, this course is an opportunity for students to make progress on their Master theses, particularly on the data analyses portion of it. For those matters, the actual content of the course will follow the students' research questions and data structure. Thus, during the course, students will perform real analyses on their own data (if they have those data already available), otherwise students will perform replications. 


\subsection*{Organization}

\begin{itemize}
  \item[{\color{red}\Pointinghand}] I need you to {\bf email/meet me two weeks before this module begins} so we can discuss about your Master thesis, particularly, (1) your research question and (2) your data structure. This will help me tailoring the actual contents of this course to your research project.
  \item[$\circ$] The course will be taught in English.
  \item[$\circ$] The course will be organized in different ``Lessons.'' Each class we will try to cover as many lessons as we can.
\end{itemize}

In terms of contents, this course will address four general topics.

\begin{enumerate}
	\item Basic functions in \texttt{R}.
	\item Descriptive statistics in \texttt{R}.
	\item Introduction to lineal models in \texttt{R}.
 	\item Causal inference in \texttt{R}.
\end{enumerate}


\subsection*{Programming}

We will learn to program in \texttt{R}, the most-used programming language in social sciences. There are several advantages. \texttt{R} is free and runs on all platforms. Second, it's an object-oriented language. This implies---third---that \texttt{R} forces the student to think hard about what s/he is doing. Unlike other statistical packages such as \texttt{Stata} or \texttt{SPSS}, where the user ``clicks and points,'' you have to tell \texttt{R} specifically \emph{what} you need and \emph{how} you need it. Fourth, if you know \texttt{R}, you can easily learn about other pieces of software.
\\
\\
{\bf Installing \texttt{R}}. First, \href{https://cloud.r-project.org}{download} \texttt{R}. Then select which version of \texttt{R} you will need depending on your OS (i.e., Windows, Mac, Ubuntu). \texttt{R} will start downloading. Once it's all done, install \texttt{R}. Now, \href{https://www.rstudio.com}{download} \texttt{R Studio}, the most-used interphase to ``talk'' to \texttt{R}. Click on \emph{Download R-Studio} and make sure you select \emph{FREE}. Also, select the version that works according to your OS.


\subsection*{Academic Integrity}

I expect nothing but the best out of my students. 

\begin{itemize}
     \item[$\circ$] I expect students to do their reading \emph{before} class.
     \item[$\circ$] Practical exercises should also be done \emph{before} class. 
     \item[$\circ$] If you need to see me, plan your time accordingly. It's best to assume that my office hours will get busier before tests and submissions. Ask your TA or myself when in doubt. 

  \item[$\circ$] I usually don't answer emails during weekends. 
\end{itemize}


\begin{itemize}
  \item[{\color{red}\Pointinghand}] Plagiarism will not be tolerated. Make sure you follow the University's rules and definitions of plagiarism. Also, make sure you know how to cite your work. 

  \item[{\color{red}\Pointinghand}] I won't accept late work.

\end{itemize}


\subsection*{Policy About Collaborative Work and External Resources}

{\bf I \emph{do} recommend collaborative work}. It's good that you work with your classmates and spend quality-time in the lab together. However, I will grade \emph{individual} work. 
\\
\\
Another advantage of \texttt{R} is that it has a really engaged community of software developers and Internet bloggers. They are your best friends whenever \texttt{R} gives you trouble. Maybe the best website to look for answers \emph{before} start asking online, is \href{https://stackoverflow.com/questions/tagged/r}{StackOverflow}. In any case, feel free to contact your TA or myself.

\subsection*{Evaluations}

\begin{enumerate}

	% Participation
	\item {\bf \emph{Quizzes}}: 5\% each, 10\% in total.
	\\
  \\
   There will be two quizzes. {\bf The first one is due at the beginning of the second session!} Both are due in hard copy (i.e. paper). Pen and paper is fine, but \LaTeX\; is best. {\bf All problems sets are due the following week \emph{before} class}.

	\item {\bf \emph{Problem Sets}}: 10\% each, 40\% in total.

The \emph{problem sets} are hands-on programming exercises. Some times, you may expect some epistemology-type questions for which I expect epistemology-type answers (one well-crafted paragraph will do). For the programming ones, I will give you an \texttt{R} script with a dataset. You'll have to answer the questions within the same \texttt{R} script, and then turn that file in. The TA and myself will be available to answer questions if needed. {\bf All problems sets are due the following week \emph{before} class}.

\begin{itemize}
		\item[$\diamond$] While it shouldn't be necessary, you \emph{may} use resources on the Internet to answer the questions.
		\item[$\diamond$] It is important that the code runs without issues. In other words, your coding shouldn't get stuck.
		\item[$\diamond$] It is important to guide and explain your reasoning. For that use the \# symbol.
\end{itemize}


\item {\bf Research Project \emph{or} Replication (30\%) plus a Final Presentation (20\%)}: 50\% in total.\\

This is the core of course. You'll have to give a 20 minutes presentation, very much like a professional conference. Using your (1) actual dataset and (2) research question of your actual Masters thesis, you will have to:

\begin{itemize}

  \item[$\diamond$] {\bf Motivate the problem}: Why should we care about your research question? (1 minute).
  \item[$\diamond$] {\bf Short lit review}: What's the main gap your work intends to bridge? (1 minute).
  \item[$\diamond$] {\bf Hypotheses}: Tell us about your hypotheses.
  \item[$\diamond$] {\bf Data}: What's your dependent and independent variables? Use plots, summary statistics, etc. (2 minutes).
  \item[$\diamond$] {\bf Data analyses and hypotheses testing}: What's the relationship between your IV and DV's? Why are you performing the analyses you're performing? Are there any alternative ways to analyze your data? \emph{Convince us you've done all that there is to be done with your data!} The audience, the TA and myself are going to ask you questions about this. Thus, anticipate those questions and make sure you include in your presentation a large Appendix covering, at least: extra diagnostics, plots, tables, alternative variable re-coding, etc. (rest of the time)
  \item[$\diamond$] {\bf Conclusion}: What are the main conclusions you can derive from your work? What are your suggestions for future research? (1 minute).
\end{itemize}

Life isn't perfect. Thus, you might not have the data ready by the time we need them for the class. In those cases, you will conduct a replication. A replication consists of obtaining someone else's data and performing the exact same data analyses published in the paper (models, tables and plots). \emph{The idea is to obtain the exact same results!} The data must come from a well-published paper. To obtain those data, you will need to contact the actual authors by sending them a \emph{very} nice email. I'll ask you to see me before sending that email. Since replications take time, you will need to gather the data of your replication ASAP. In any case, {\bf you will need to have the data ready from our second meeting}. {\bf \color{blue}Failure to obtain the replication data will negatively impact your grade tremendously}.
\\
\\
Regardless, either you have your own data ready or you're conducting a replication, you'll have to:

\begin{itemize}
  \item[$\diamond$] {\bf Email/meet me one week before this module begins} so we work out a plan. We will discuss whether you having your Master thesis dataset is possible at all, otherwise, which paper you will replicate. Hopefully is a paper that either addresses a similar problem of your thesis, or it's a paper you consider the best paper in your area. 
  \item[$\diamond$] Turn in an \texttt{R} script with \emph{all} data manipulations.
  \item[$\diamond$] Give the 20-minutes presentation at the end of the module.
\end{itemize}

\end{enumerate}


\underline{To summarize}:

\begin{table}[H]
\centering
\begin{tabular}{ccc}
						  	& \textbf{Percentage} & {\bf Cumulative Percentage} \\
							\hline
Quizz \#1      	& 5\%       	 & 5\% \\
Quizz \#2       & 5\%         & 10\% \\
\hline
Problem Set \#1 													 & 10\% 		 & 20\%  \\
Problem Set \#2 													 & 10\% 		 & 30\%  \\
Problem Set \#3 													 & 10\% 		 & 40\%  \\
Problem Set \#4 													 & 10\% 		 & 50\%  \\
\hline
Research Project or Replication						 & 30\% 	 	 & 80\% \\
Final Presentation												 & 20\% 	 	 & 100\% \\
\hline             
\end{tabular}
\end{table}


\subsection*{Basic Bibliography}

\begin{itemize}
  \item[$\bullet$] \fullcite{Imbens2015}.
  \item[$\bullet$] \fullcite{Angrist2009}.
  \item[$\bullet$] \fullcite{Wooldridge2002}.
  \item[$\bullet$] \fullcite{Namboodiri1984}.
  \item[$\bullet$] \fullcite{Gill:2006wp}.
  \item[$\bullet$] \fullcite{Fox:2010vc}

\end{itemize}

\subsection*{Suggested Bibliography}

\begin{itemize}
  \item[$\bullet$] \fullcite{Rosenbaum2010a}.
  \item[$\bullet$] \fullcite{Monogan2015}.
\end{itemize}


\begin{itemize}
\item[{\color{red}\Pointinghand}] We will also read some papers.
\end{itemize}


\subsection*{Schedule}


\begin{enumerate}
	\item {\color{ForestGreen}{\bf Basic Functions in \texttt{R}}}

			\begin{itemize} 
				\item[$\bullet$] {\bf Lesson \# 1} % original Lesson \#1
				\begin{itemize} 
					\item[$\circ$] Introduction: syllabus, requirements, expectations, etc.
					\item[$\circ$] \emph{What's \texttt{R}?} Installing \texttt{R} and \texttt{RStudio}.
          %\item[$\circ$] \emph{Qu\'e es \texttt{Stata}?}
					\item[$\circ$] {\bf Reading(s)}: 
						\begin{itemize} 
							\item[$\diamond$] \textcite{Wooldridge2002}: 1.
									\end{itemize}
					% Monogan2015 Ch. 1
				\end{itemize}
			\end{itemize}






			\begin{itemize} 
				\item[$\bullet$] {\bf Lesson \# 2} % original Lesson \#2
				\begin{itemize} 
					\item[$\circ$] Basic functions: mean, \texttt{help()}, operators, objects (character, arrays, dates, lists, data.frames).
					\item[$\circ$] Working with data.frames (I): format, labels, types of variables, basic description. % ch 2 fox r companion, Monogan2015 Ch 2, https://stats.idre.ucla.edu/stat/data/intro_r/intro_r.html#(12)
					\item[$\circ$] {\bf Reading(s)}: 
					\begin{itemize}
						\item[$\diamond$] \textcite{Fox:2010vc}: 1.1.
					\end{itemize}
				\end{itemize}
			\end{itemize}




			\begin{itemize} 
				\item[$\bullet$] {\bf Lesson \# 3} % original Lesson \#3
					\begin{itemize} 
				\item[$\circ$] Transformations, generating new variables.
				\item[$\circ$] Working with data.frames (II): generating matrices and data.frames, \texttt{merge}, \texttt{append}. Logs.  %(p. 35 Gill, Essential Math)
        \item[$\circ$] {\bf Reading(s)}: 
          \begin{itemize}
            \item[$\diamond$] \textcite{Gill:2006wp}: 1.7.
            \item[$\diamond$] \textcite{Fox:2010vc}: 2.3 and 3.4.
          \end{itemize}
					\end{itemize}
					% Monogan2015 Ch 2, 
			\end{itemize}




			\begin{itemize} 
				\item[$\bullet$] {\bf Lesson \# 4} % original Lesson \#4
					\begin{itemize} 
						\item[$\circ$] Data visualization (I): bar plots, scatter plots, histograms, time series plots. % fox Applied: ch. 3, fox companion ch's 3.1-3.3, Monogan2015 Ch 3
						\item[$\circ$] {\bf Reading(s)}:
							\begin{itemize}
								\item[$\diamond$] \textcite{Fox:2010vc}: 3.2.
							\end{itemize}
					\end{itemize}
			\end{itemize}



			%\begin{itemize} 
			%	\item[$\bullet$] {\bf Lesson \\#5}
			%		\begin{itemize} 
			%	\item[$\circ$] Data visualization (II): more complex plots, maps.
				% Monogan2015 Ch 3, fox ch 3
			%	\item[$\circ$] {\bf Reading(s)}: 
			%		\begin{itemize}
			%			\item[$\diamond$] \textcite{Fox:2010vc}: 7.3.
			%		\end{itemize}
			%		\end{itemize}
			%\end{itemize}


	\item {\color{ForestGreen}{\bf Descriptive Statistics in \texttt{R}}}

			\begin{itemize} 
				\item[$\bullet$] {\bf Lesson \# 5} % original Lesson \#6
					\begin{itemize} 
				\item[$\circ$] Descriptive Statistics (I). Measures of central tendency (mean, median, mode) and dispersion (variance and standard deviation). % sy_1/week 5, Monogan2015 Ch 4, Fox_Appendices (App D), Chs. 8  (Gill, Essential Math)
          \item[$\circ$] {\bf Reading(s)}: 
          \begin{itemize}
            \item[$\diamond$] \textcite{Gill:2006wp}: 8.4---8.5.
          \end{itemize}
					\end{itemize}
			\end{itemize}


			%\begin{itemize} 
			%	\item[$\bullet$] {\bf Lesson \\#7}
			%		\begin{itemize} 
			%	\item[$\circ$] Descriptive Statistics (II). Probability Theory and distributions: binomial, normal, others; simulation. % sy_1/week 6, fox applie ch. 4, Fox_Appendices (App D)., Chs. 8 (Gill, Essential Math).
       %  \item[$\circ$] {\bf Reading(s)}: 
        %  \begin{itemize}
         %   \item[$\diamond$] \textcite{Gill:2006wp}: 8.3.
          %\end{itemize}
					%\end{itemize}
			%\end{itemize}


%\item[{\color{red}\Pointinghand}] Problem set \#1. Turn it in next class.


	\item {\color{ForestGreen}{\bf Introduction to Linear Regression in \texttt{R}}}


			\begin{itemize} 
				\item[$\bullet$] {\bf Lesson \# 6} % original Lesson \#8
					\begin{itemize} 
						\item[$\circ$] \emph{What's OLS?}
						\item[$\circ$] {\bf Reading(s)}: 
							\begin{itemize}
								\item[$\diamond$] \textcite{Wooldridge2002}: 2.1---2.2.
								% fox ch 5
							\end{itemize}
					\end{itemize}
			\end{itemize}



			%\begin{itemize} 
			%	\item[$\bullet$] {\bf }
			%		\begin{itemize} 
			%			\item[$\circ$] La mec\'anica detr\'as del OLS (I): matrices ``a mano''.
			%			\item[$\circ$] {\bf Reading(s)}: 
			%				\begin{itemize}
			%					\item[$\diamond$] \textcite{Namboodiri1984}: Ch. 1 y 2.
						% Monogan2015 Ch. 10.3.1
			%				\end{itemize}
			%		\end{itemize}
			%\end{itemize}

			\begin{itemize} 
				\item[$\bullet$] {\bf Lesson \# 7} % original Lesson \#9
					\begin{itemize} 
						\item[$\circ$] The mechanics behind OLS (II): matrices in \texttt{R}.
            \item[$\circ$] {\bf Reading(s)}: 
            \begin{itemize}
              \item[$\diamond$] \textcite{Namboodiri1984}: 1---2.
            \end{itemize}
% Monogan2015 Ch. 10.3.2
					\end{itemize}
			\end{itemize}



			\begin{itemize} 
				\item[$\bullet$] {\bf Lesson \# 8} % original Lesson \#10
					\begin{itemize} 
						\item[$\circ$] Coefficients. % ex's C2.1 Wooldridge.
						\item[$\circ$] {\bf Reading(s)}: 
							\begin{itemize}
								\item[$\diamond$] \textcite{Wooldridge2002}: 3.1---3.2.
                \item[$\diamond$] \textcite{Fox:2010vc}: 4.3 till p. 177.
								% como interpretar los coeficientes, fox & weisberg 4.3 till p 177
							\end{itemize}
					\end{itemize}
			\end{itemize}



			\begin{itemize} 
				\item[$\bullet$] {\bf Lesson \# 9} % original Lesson \#11
					\begin{itemize} 
						\item[$\circ$] Error, residual and $\epsilon_{i}$: Statistical, practical and philosophical differences (and why it matters).
					\end{itemize}
			\end{itemize}


			\begin{itemize} 
				\item[$\bullet$] {\bf Lesson \# 10} % original Lesson \#12
					\begin{itemize} 
						\item[$\circ$] Confidence intervals, standard error and covariance matrix. % sy_1 W8
						\item[$\circ$] {\bf Reading(s)}: 
							\begin{itemize}
								\item[$\diamond$] \textcite{Wooldridge2002}: 4.3.
                \item[$\diamond$] \textcite{Fox:2010vc}: 4.3.1.
								% Fox & weisberg 4.3.1, fox 6.1.3, montgomery 2.4
							\end{itemize}
					\end{itemize}
			\end{itemize}



			\begin{itemize} 
				\item[$\bullet$] {\bf Lesson \# 11} % original Lesson \#13
					\begin{itemize} 
						\item[$\circ$] Hypothesis testing (t test), Type I and II Errors, statistical significance (p-values). 
						\item[$\circ$] {\bf Reading(s)}: 
							\begin{itemize}
								\item[$\diamond$] \textcite{Wooldridge2002}: 4.2.
							\end{itemize}
					\end{itemize}
			\end{itemize}



			%\begin{itemize} 
			%	\item[$\bullet$] {\bf Lesson \\#14}
			%		\begin{itemize} 
			%			\item[$\circ$] Interaction terms. Motivation. Estimation. Interpretation.  
			%			\item[$\circ$] {\bf Reading(s)}: 
			%				\begin{itemize}
			%					\item[$\diamond$] \textcite{Wooldridge2002}: 7.4.
			%					\item[$\diamond$] \fullcite{Brambor2006}.
			%				\end{itemize}
			%		\end{itemize}
			%\end{itemize}

%\item[{\color{red}\Pointinghand}] Problem set \#2. Turn it in next class.


			\begin{itemize} 
				\item[$\bullet$] {\bf Lesson \# 12} % original Lesson \#15
					\begin{itemize} 
						\item[$\circ$] Numeric properties of OLS, Gauss-Markov, omitted variable bias. % sy_1 W10
						\item[$\circ$] {\bf Reading(s)}: 
							\begin{itemize} 
								\item[$\diamond$] \textcite{Wooldridge2002}: pp. 89---94, 102---104.
							\end{itemize}
					\end{itemize}
			\end{itemize}



			\begin{itemize} 
				\item[$\bullet$] {\bf Lesson \# 13} % original Lesson \#16
					\begin{itemize} 
						\item[$\circ$] Goodness of fit, coefficient of determination (r$^2$), prediction ($\hat y$). 
						\item[$\circ$] {\bf Reading(s)}:
							\begin{itemize} 
								\item[$\diamond$] \textcite{Wooldridge2002}: pp. 40---41, 6.3.
								\item[$\diamond$] \fullcite{King1986}.

							\end{itemize}
					\end{itemize}
			\end{itemize}



			\begin{itemize} 
				\item[$\bullet$] {\bf Lesson \# 14} % original Lesson \#17
					\begin{itemize} 
						\item[$\circ$] Issues and post-estimation: perfect multicollinearity and variance inflation, heteroskedasticity, non-linearity, outliers, non-normality of residuals and auto-correlation. % sy_1 W14, W15, explicar que multicol causa VARIANCE INFLATION (matrix form),
						\item[$\circ$] {\bf Reading(s)}: 
						\begin{itemize}
						\item[$\diamond$] \textcite{Wooldridge2002}: 8 y 9.5.
						\end{itemize}
					\end{itemize}
			\end{itemize}



      % \begin{itemize} 
      %  \item[$\bullet$] {\bf Clase}
      %    \begin{itemize} 
      %      \item[$\circ$] Presentaci\'on de resultados: tablas, gr\'aficos. Variables independientes categoricas.
      %      \item[$\circ$] {\bf Reading(s)}: 
      %        \begin{itemize}
      %          \item[$\diamond$] Jeffrey Wooldridge, 2010. \href{https://github.com/hbahamonde/OLS/raw/master/Readings/Wooldridge.pdf}{\emph{Introducci\'on a la Econometr\'ia. Un Enfoque Moderno}}. Cengage Learning: Ch. 4.6.\phantom{\textcite{Wooldridge2002}}
      %        \end{itemize}
      %  {\color{orange}\item[$\bigstar$] Entrega del temario para la tarea pr\'actica ``grande'': \#3}.
      %    \end{itemize}
      % \end{itemize}

%\item[{\color{red}\Pointinghand}] Problem set \#3. Turn it in next class.


  \item {\color{ForestGreen}{\bf Causal Inference in \texttt{R}}} (if time permits)


      \begin{itemize} 
        \item[$\bullet$] {\bf Lesson \# 15} % original Lesson \#18
          \begin{itemize} 
            \item[$\circ$] Causal Inference: The Fundamental Problem of Causal Inference, the ignorability assumption, and the Potential Outcomes Framework.
            \item[$\circ$] {\bf Reading(s)}: 
              \begin{itemize}
                \item[$\diamond$] \textcite{Imbens2015}: 1.
              \end{itemize}
          \end{itemize}
      \end{itemize}





      \begin{itemize} 
        \item[$\bullet$] {\bf Lesson \# 16} % original Lesson \#19
          \begin{itemize} 
            \item[$\circ$] Instrumental Variables and the Two-Stage Least Squares Regression.
            \item[$\circ$] {\bf Reading(s)}: 
              \begin{itemize}
                \item[$\diamond$] \textcite{Angrist2009}: 4.1---4.2.
              \end{itemize}
          \end{itemize}
      \end{itemize}


%\item[{\color{red}\Pointinghand}] Problem set \#4. Turn it in next class.


      \begin{itemize} 
        \item[$\bullet$] {\bf Lesson \# 17} % original Lesson \#20
          \begin{itemize} 
            \item[$\circ$] Regression Discontinuity Designs (RDD): \emph{Sharp Designs}.
            \item[$\circ$] {\bf Reading(s)}: 
              \begin{itemize}
                \item[$\diamond$] \textcite{Angrist2009}: 6---6.1.
              \end{itemize}
          \end{itemize}
      \end{itemize}


      \begin{itemize} 
        \item[$\bullet$] {\bf Lesson \# 18} % original Lesson \#21
          \begin{itemize} 
            \item[$\circ$] Regression Discontinuity Designs (RDD): \emph{Fuzzy Designs}.
            \item[$\circ$] {\bf Reading(s)}: 
              \begin{itemize}
                \item[$\diamond$] \textcite{Angrist2009}: 6.2.
              \end{itemize}
          \end{itemize}
      \end{itemize}


      \begin{itemize} 
        \item[$\bullet$] {\bf Lesson \# 19} % original Lesson \#22
          \begin{itemize} 
            \item[$\circ$] Incorporating the Time Element: Fixed Effects (FE), Differences-in-Differences (DID).
            \item[$\circ$] {\bf Reading(s)}: 
              \begin{itemize}
                \item[$\diamond$] \textcite{Angrist2009}: 5.
              \end{itemize}
          \end{itemize}
      \end{itemize}



      \begin{itemize} 
        \item[$\bullet$] {\bf Last Class}
          \begin{itemize} 
            \item[$\circ$] Presentations. Format is conference.
        \end{itemize}
      \end{itemize}


			

\end{enumerate}


\newpage
\pagenumbering{roman}
\setcounter{page}{1}
\printbibliography



\end{document}


