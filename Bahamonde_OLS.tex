%LaTeX Curriculum Vitae Template
%
% Copyright (C) 2004-2009 Jason Blevins <jrblevin@sdf.lonestar.org>
% http://jblevins.org/projects/cv-template/
%
% You may use use this document as a template to create your own CV
% and you may redistribute the source code freely. No attribution is
% required in any resulting documents. I do ask that you please leave
% this notice and the above URL in the source code if you choose to
% redistribute this file.

\documentclass[letterpaper]{article}

\usepackage{hyperref}
\hypersetup{
    bookmarks=true,         % show bookmarks bar?
    unicode=false,          % non-Latin characters in Acrobat’s bookmarks
    pdftoolbar=true,        % show Acrobat’s toolbar?
    pdfmenubar=true,        % show Acrobat’s menu?
    pdffitwindow=true,     % window fit to page when opened
    pdfstartview={FitH},    % fits the width of the page to the window
    pdftitle={My title},    % title
    pdfauthor={Author},     % author
    pdfsubject={Subject},   % subject of the document
    pdfcreator={Creator},   % creator of the document
    pdfproducer={Producer}, % producer of the document
    pdfkeywords={keyword1} {key2} {key3}, % list of keywords
    pdfnewwindow=true,      % links in new window
    colorlinks=true,       % false: boxed links; true: colored links
    linkcolor=blue,          % color of internal links (change box color with linkbordercolor)
    citecolor=blue,        % color of links to bibliography
    filecolor=blue,      % color of file links
    urlcolor=blue           % color of external links
}



\usepackage{geometry}
\usepackage{import} % To import email.
\usepackage{marvosym} % face package
%\usepackage{xcolor,color}
\usepackage{fontawesome}
\usepackage{amssymb} % for bigstar
\usepackage{epigraph}
\usepackage{float}
\usepackage[svgnames]{xcolor}

% Comment the following lines to use the default Computer Modern font
% instead of the Palatino font provided by the mathpazo package.
% Remove the 'osf' bit if you don't like the old style figures.
\usepackage[T1]{fontenc}
\usepackage[sc,osf]{mathpazo}

% Set your name here
\def\name{Quantitative Module}

% Replace this with a link to your CV if you like, or set it empty
% (as in \def\footerlink{}) to remove the link in the footer:
\def\footerlink{}
% \href{http://www.hectorbahamonde.com}{www.HectorBahamonde.com}

% The following metadata will show up in the PDF properties
\hypersetup{
  colorlinks = true,
  urlcolor = blue,
  pdfauthor = {\name},
  pdfkeywords = {intro to social sciences},
  pdftitle = {\name: Syllabus},
  pdfsubject = {Syllabus},
  pdfpagemode = UseNone
}

\geometry{
  body={6.5in, 8.5in},
  left=1.0in,
  top=1.25in
}

% Customize page headers
\pagestyle{myheadings}
\markright{{\tiny \name}}
\thispagestyle{empty}

% Custom section fonts
\usepackage{sectsty}
\sectionfont{\rmfamily\mdseries\Large}
\subsectionfont{\rmfamily\mdseries\itshape\large}

% Don't indent paragraphs.
\setlength\parindent{0em}

% Make lists without bullets
\renewenvironment{itemize}{
  \begin{list}{}{
    \setlength{\leftmargin}{1.5em}
  }
}{
  \end{list}
}


%%% bib begin
\usepackage[american]{babel}
\usepackage{csquotes}
%\usepackage[style=chicago-authordate,doi=false,isbn=false,url=false,eprint=false]{biblatex}

\usepackage[authordate,isbn=false,doi=false,url=false,eprint=false]{biblatex-chicago}
\DeclareFieldFormat[article]{title}{\mkbibquote{#1}} % make article titles in quotes
\DeclareFieldFormat[thesis]{title}{\mkbibemph{#1}} % make theses italics

\AtEveryBibitem{\clearfield{month}}
\AtEveryCitekey{\clearfield{month}}

\addbibresource{/Users/hectorbahamonde/Bibliografia_PoliSci/library.bib} 
\addbibresource{/Users/hectorbahamonde/Bibliografia_PoliSci/Bahamonde_BibTex2013.bib} 

% USAGES
%% use \textcite to cite normal
%% \parencite to cite in parentheses
%% \footcite to cite in footnote
%% the default can be modified in autocite=FOO, footnote, for ex. 
%%% bib end




\begin{document}

% Place name at left
%{\huge \name}

% Alternatively, print name centered and bold:
\centerline{\huge \bf \name}

\epigraph{\emph{Statistics: ``science dealing with data about the condition of a state or community''}}{Gottfried Aschenwall, 1770}


\vspace{0.25in}

\begin{minipage}{0.45\linewidth}
 University of Turku \\
  Faculty of Social Sciences \\
  Turku, Finland\\
  \\
  \\

\end{minipage}
\hspace{4cm}\begin{minipage}{0.45\linewidth}
  \begin{tabular}{ll}
{\bf Last updated}: \today. \\
 {\bf Download last version} \href{https://github.com/hbahamonde/OLS/raw/master/Bahamonde_OLS.pdf}{here}.%\\
   %{\bf {\color{red}{\scriptsize Not intended as a definitive version}}} %\\
    \\
    \\
    \\
    \\
    \\
  \end{tabular}
\end{minipage}

\subsection*{General Overview}


\vspace{1mm}
{\bf Professor}: H\'ector Bahamonde, PhD.\\
\texttt{e:}\href{mailto:hibano@utu.fi}{\texttt{hibano@utu.fi}}\\
\texttt{w:}\href{http://www.hectorbahamonde.com}{\texttt{www.HectorBahamonde.com}}\\
{\bf Office Hours}: Schedule time with me \href{https://calendly.com/bahamonde}{\texttt{here}}.\\

\vspace{5mm}
{\bf Place}: TBA.\\
{\bf Time}: TBA.\\

{\bf Course website}: \href{http://www.hectorbahamonde.com}{TBA}.

\vspace{5mm}
{\bf (TA)}: TBA.\\
\texttt{e:}: TBA.\\
{\bf TA Bio}: TBA.\\
{\bf Hora de ayudant\'ia}: \emph{TBA}.\\


\vspace{5mm}
{\bf Program}:  Master of Social Sciences, University of Turku.\\
{\bf Semester}: Spring.\\
{\bf Credits}: 2.\\
{\bf Timing}: 4 modules of 3 hours each. \\





\subsection*{Motivation: Why take this course?}

\emph{What's the effect of education on income? How can we evaluate a public policy? Does legalizing some drug increase consumption? Which political candidate will win the election?} 
\\
\\
Public entities guide their strategic decisions based on quantifiable information, i.e., data. This decision-making process has taken even much more relevance nowadays where there has been a wave of data digitization, making available much more quality data. This is fundamental for social scientists like us, putting heavy pressures for us to learn how to analyze those data. If ten years ago we complained that there was not enough data, our problem now is different: there are so much data that we need to learn how to analyze it.
\\
\\
Though what we will learn this semester is highly mathematical and numeric, don't get confused: these methods are not ``bullet-proof,'' they will \emph{never} ``proof'' anything at all. During the semester I need you to exercise your criticism skills all the time. As we will see, you will need to \emph{suspect} everything we do, particularly, because we will learn \emph{inferential statistics}, that is, we will do \emph{inferences}, not ``proofs.'' On top of that, everything we do will come with some degree of \emph{\bf uncertainty} and hard-to-check \emph{\bf assumptions}, probably the most-used words throughout the semester. 
\\
\\
Depending on our progress during the semester, we will pay special attention to an issue that is absolutely relevant nowadays in applied social sciences: \emph{causal inference}. For those matters, we will discuss why the methods we will learn this semester are not causal, i.e., we cannot derive a causal statement from our results. However, when certain conditions apply, we might get \emph{quasi-causal} statements. What are those ``conditions''? We shall see...
\\
\\
Honestly, I hope this course captivates your enthusiasm, and gets you interested and curious about ways to study social phenomena, \emph{tervetuloa}!


\subsection*{Description}

Enrolled students will acquire a basic inferential statistics toolkit. The course will pay special attention to Ordinary Least Square regression (OLS) and a selection of Generalised Linear Models (such as logit/probit, multinomial, ordered models and/or rare events data generating processes)---the workhorses of quantitative social sciences. This course is very hands-on, and while some statistical theory will be covered, the core of it will be on data analyses and programming in \texttt{R}. 
\\
\\
Overall, this course is an opportunity for students to make progress on their Master theses, particularly on the data analyses portion of it. For those matters, the actual content of the course will follow the students' research questions and data structure. Thus, during the course, students will perform real analyses on their own data (if they have those data already available), otherwise students will perform replications. 

\subsection*{Organization}

\begin{itemize}
  \item[$\circ$] The course will be taught in English in the computer lab as 4 sessions of 3 hours each.
  \item[$\circ$] We will meet between early March and early May at the end of the week (Wed, Thu and/or Fri) starting at 11.00 AM. Exact dates will be confirmed later.
\end{itemize}

In terms of contents, this course will address four general topics.

\begin{enumerate}
	\item Basic functions in \texttt{R}.
	\item Descriptive statistics in \texttt{R}.
	\item Introduction to lineal models in \texttt{R}.
 	\item Causal inference in \texttt{R}.
\end{enumerate}


\subsection*{Programming}

We will learn to program in \texttt{R}, the most-used programming language in social sciences. There are several advantages. \texttt{R} is free and runs on all platforms. Second, it's an object-oriented language. This implies---third---that \texttt{R} forces the student to think hard about what s/he is doing. Unlike other statistical packages such as \texttt{Stata} or \texttt{SPSS}, where the use ``clicks and points,'' you have to tell \texttt{R} specifically what you need and how you need it. Fourth, if you know \texttt{R}, you can easily learn about other pieces of software.

{\bf Installing \texttt{R}}. First, install \texttt{R} from the \href{https://www.r-project.org/}{official Website}. Click on ``CRAN'' (upper-left corner), then select any ``mirror'' you want. \texttt{R} will start downloading. Once it's all done, install \texttt{R}. Now, download \texttt{R Studio}, the most-used interphase to ``talk'' to \texttt{R}. For those purposes, \href{https://www.rstudio.com}{download} this piece of software from its official website. Click on \emph{Download R-Studio} and make sure you select \emph{FREE}. Also, select the version that works according to your OS (i.e., Windows, Mac, Ubuntu).


\subsection*{Academic Integrity}

I expect nothing but the best out of my students, in particular, it's necessary to mention the following:

\begin{itemize}
     \item[$\circ$] I expect students to do their reading \emph{before} class. Participation is not only encouraged but \emph{graded}.
     \item[$\circ$] Practical exercises should also be done \emph{before} class. 
     \item[$\circ$] If you need to see me, plan your time accordingly. It's good to assume that it will get busier before tests and submissions. 

  \item[$\circ$] I usually don't answer emails during weekends. 
\end{itemize}


\begin{itemize}
  \item[{\color{red}\Pointinghand}] Plagiarism will not be tolerated. Make sure you follow the University's rules and definitions of plagiarism. Also, make sure you know how to cite your work. 

  \item[{\color{red}\Pointinghand}] I won't accept late work.

\end{itemize}


\subsection*{Policy About Collaborative Work}

{\bf I do recommend collaborative work}. It's good that you work with your classmates. However, I will grade individual work. 

\subsection*{Recitation}

TBA.



\subsection*{Evaluations}

\begin{enumerate}

	% Participation
	\item {\bf Lecturas y Participaci\'on }: 10\%.
	
    El TA y yo asumiremos durante todo el semestre que haz le\'ido. Nosotros empleamos un m\'etodo de clases interactivo, pero este m\'etodo necesita de tu participaci\'on activa en clases.
    \\
    \\  
    Si no puedes asistir a la clase sincr\'onica, existir\'an opciones para dejar entradas en la secci\'on \emph{Foro} de \texttt{uCampus}.

	\item {\bf \emph{Problem Sets}}: 10\% cada uno, 40\% en total.

Estos \emph{problem sets} son ejercicios pr\'acticos. Nosotros te entregaremos un \emph{script} de \texttt{R} junto a una base de datos. T\'u tendr\'as que resolver las preguntas dentro de \texttt{R} y devolvernos ese \emph{script}. El ayudante y el profesor estaran disponibles para resolver preguntas v\'ia email o Zoom.

\begin{itemize}
		\item[$\diamond$] {\bf Aunque no es necesario, s\'i puedes ocupar recursos externos, como Internet}.
		\item[$\diamond$] Es importante que estas l\'ineas corran bien: el usuario (yo) tiene que ser cap\'az de ver c\'omo \texttt{R} ejecuta cada linea, sin estancarse.
		\item[$\diamond$] Es importante que vayas guiando al usuario (yo) sobre tu raciocinio. Aseg\'urate de comentar (usando el simbolo \#).
\end{itemize}


\item {\bf Un trabajo final obligatorio/no-eximible (30\%) y una presentaci\'on final (20\%, v\'ia Zoom)}: 50\% en total.\\


En este curso, la actividad final es un trabajo final (30\%) que tiene formato de trabajo grupal. Usando una base de datos que nosotros te daremos, t\'u y tu grupo deber\'an responder una serie de preguntas. El producto final (i.e. lo que debes entregar) consiste en un \emph{script} de \texttt{R}. La nota es grupal (i.e. todo el grupo recibir\'a la misma nota). {\bf Los grupos ser\'an de 2 personas}. La formacion del grupo es end\'ogena.
\\
\\
El paper (\emph{script}) se puede entregar antes, pero una vez cerrado el plazo, no se recibir\'an trabajos. Los \emph{scripts} que se entreguen tarde o v\'ia \emph{email} tendr\'an un 1 (sin opci\'on a reclamo). {\bf No hay excepciones}. 
\\
\\
En un formato muy parecido a una conferencia acad\'emica (virtual, no presencial), tendr\'as (junto a tu grupo) que presentar los principales hallazgos (20\%). Todos/as presentan. Cada presentaci\'on debe durar no menos de 15 minutos, pero nunca m\'as de 20 minutos. Las presentaciones se realizar\'an virtualmente (i.e. v\'ia Zoom) el \'ultimo d\'ia de clases. Tendr\'as que ocupar \emph{slides} (``Power Point''). Para tales efectos, tendr\'as que compartir pantalla desde tu casa, y hacer tu presentaci\'on de esa manera.



%\\
%\\
Les recomiendo ``verme'' (v\'ia Zoom) en \href{https://calendly.com/bahamonde/officehours}{mis office hours} \emph{antes} del plazo de entrega. Si quieres, SEND ME AN EMAIL con tu borrador, y yo te devolver\'e comentarios. V\'elo como una pre-correcci\'on. Esto es voluntario. Tambi\'en puedes contactar al/la TA. {\bf No se procesar\'an preguntas durante fines de semana, y/o festivos}.


\end{enumerate}


\underline{En resumen}:

\begin{table}[H]
\centering
\begin{tabular}{ccc}
							& \textbf{Porcentaje} & {\bf Porcentaje Acumulado} \\
							\hline
Participaci\'on (c\'atedra, foro \texttt{uCampus} y ayudant\'ia) 	 & 10\%       	 & 10\% \\
\hline
\emph{Problem Set} \#1 													 & 10\% 		 & 20\%  \\
\emph{Problem Set} \#2 													 & 10\% 		 & 30\%  \\
\emph{Problem Set} \#3 													 & 10\% 		 & 40\%  \\
\emph{Problem Set} \#4 													 & 10\% 		 & 50\%  \\
\hline
Trabajo final grupal 												 & 30\% 	 	 & 80\% \\
Presentaci\'on grupal												 & 20\% 	 	 & 100\% \\
\hline             
\end{tabular}
\end{table}


\subsection*{Textos M\'inimos}

\begin{itemize}
  \item[$\bullet$] Guido Imbens and Donald Rubin (1998). \href{https://github.com/hbahamonde/OLS/raw/master/Readings/Imbens_Rubin.pdf}{\emph{Causal Inference for Statistics, Social, and Biomedical Sciences}}.\phantom{\textcite{Imbens2015}}
  \item[$\bullet$] Joshua Angrist and Jorn-Steffen Pischke (2009). \href{https://github.com/hbahamonde/OLS/raw/master/Readings/MHE.pdf}{\emph{Mostly Harmless Econometrics: An Empiricist's Companion}}.\phantom{\textcite{Angrist2009}}
  \item[$\bullet$] Jeffrey Wooldridge (2010). \href{https://github.com/hbahamonde/OLS/raw/master/Readings/Wooldridge.pdf}{\emph{Introducci\'on a la Econometr\'ia. Un Enfoque Moderno}}.\phantom{\textcite{Wooldridge2010}}
  \item[$\bullet$] Urdinez y Cruz (2019). \href{https://arcruz0.github.io/libroadp/index.html}{\emph{AnalizaR Datos Pol\'iticos}}.\phantom{\textcite{Urdinez:2019aa}}
  \item[$\bullet$] Krishnan Namboodiri (1984). \href{https://github.com/hbahamonde/OLS/raw/master/Readings/Namboodiri.pdf}{\emph{Matrix Algebra, an Introduction}}.\phantom{\textcite{Namboodiri1984}}

\end{itemize}

\subsection*{Textos Recomendados}

\begin{itemize}
  \item[$\bullet$] Paul Rosenbaum (2010). \href{https://github.com/hbahamonde/OLS/raw/master/Readings/Rosenbaum.pdf}{\emph{Design of Observational Studies}}.\phantom{\textcite{Rosenbaum2010a}}
  \item[$\bullet$] James Monogan (2015). \href{https://github.com/hbahamonde/OLS/raw/master/Readings/Monogan.pdf}{\emph{Political Analysis Using R}}.\phantom{\textcite{Monogan2015}}
\end{itemize}


\begin{itemize}
\item[{\color{red}\Pointinghand}] Tambi\'en se considerar\'an algunos \emph{papers}. Estos estar\'an se\~nalados en las fechas indicadas y en la secci\'on de Bibliograf\'ia.
\end{itemize}


\subsection*{Calendario}


\begin{enumerate}
	\item {\color{ForestGreen}{\bf Funciones b\'asicas en \texttt{R}}}

			\begin{itemize} 
				\item[$\bullet$] {\bf Clase \#1}
				\begin{itemize} 
					\item[$\circ$] Introducciones: programa de curso, requerimientos, expectativas, etc.
					\item[$\circ$] \emph{Qu\'e es \texttt{R}?} Instalaci\'on de \texttt{R} y \texttt{RStudio}.
          %\item[$\circ$] \emph{Qu\'e es \texttt{Stata}?}
					\item[$\circ$] {\bf Lecturas}: 
						\begin{itemize} 
							\item[$\diamond$] \textcite{Wooldridge2010}: Cap. 1.
							\item[$\diamond$] \textcite{Urdinez:2019aa}: Cap. 2.
						\end{itemize}
					% Monogan2015 Ch. 1
				\end{itemize}
			\end{itemize}






			\begin{itemize} 
				\item[$\bullet$] {\bf Clase \#2}
				\begin{itemize} 
					\item[$\circ$] Funciones b\'asicas: promedio, \texttt{help()}, operadores, tipos de objetos (\emph{character}, \emph{arrays}, fechas, listas, \emph{dataframes}).
					\item[$\circ$] Cargando bases de datos (I): formatos, etiquetas, tipos de variables, descripci\'on b\'asica. % ch 2 fox r companion, Monogan2015 Ch 2, https://stats.idre.ucla.edu/stat/data/intro_r/intro_r.html#(12)
					\item[$\circ$] {\bf Lecturas}: 
					\begin{itemize}
						\item[$\diamond$] \textcite{Urdinez:2019aa}: Cap. 5.
					\end{itemize}
				\end{itemize}
			\end{itemize}




			\begin{itemize} 
				\item[$\bullet$] {\bf Clase \#3}
					\begin{itemize} 
				\item[$\circ$] Cargando bases de datos (II): transformaciones, creaci\'on de nuevas variables.
				\item[$\circ$] Manipulando bases de datos: generaci\'on de matrices y \emph{dataframes}, \texttt{merge}, \texttt{append}. Logs.  %(p. 35 Gill, Essential Math)
					\end{itemize}
					% Monogan2015 Ch 2, 
			\end{itemize}




			\begin{itemize} 
				\item[$\bullet$] {\bf Clase \#4}
					\begin{itemize} 
						\item[$\circ$] Visualizaci\'on de datos (I): \emph{bar plots} (variable categ\'orica/continua, categ\'orica/categ\'orica), \emph{scatter plots}, histogramas, \emph{time series plots}. % fox Applied: ch. 3, fox companion ch's 3.1-3.3, Monogan2015 Ch 3
						\item[$\circ$] {\bf Lecturas}:
							\begin{itemize}
								\item[$\diamond$] \textcite{Urdinez:2019aa}: Cap. 4.
							\end{itemize}
					\end{itemize}
			\end{itemize}



			\begin{itemize} 
				\item[$\bullet$] {\bf Clase \#5}
					\begin{itemize} 
				\item[$\circ$] Visualizaci\'on de datos (II): \emph{plots} m\'as complejos (por categor\'ias), mapas.
				% Monogan2015 Ch 3, fox ch 3
				\item[$\circ$] {\bf Lecturas}: 
					\begin{itemize}
						\item[$\diamond$] \textcite{Urdinez:2019aa}: Cap. 15.
					\end{itemize}
					\end{itemize}
			\end{itemize}


	\item {\color{ForestGreen}{\bf Estad\'istica descriptiva en \texttt{R}}}

			\begin{itemize} 
				\item[$\bullet$] {\bf Clase \#6}
					\begin{itemize} 
				\item[$\circ$] Estad\'istica descriptiva (I): Teor\'ia de probabilidades: distribuciones, varianza. % sy_1/week 5, Monogan2015 Ch 4, Fox_Appendices (App D), Chs. 7  (Gill, Essential Math)
					\end{itemize}
			\end{itemize}


			\begin{itemize} 
				\item[$\bullet$] {\bf Clase \#7}
					\begin{itemize} 
				\item[$\circ$] Estad\'istica descriptiva (II): binomial, normal, otras; simulaci\'on. % sy_1/week 6, fox applie ch. 4, Fox_Appendices (App D)., Chs. 8 (Gill, Essential Math).
					\end{itemize}
			\end{itemize}


\item[{\color{red}\Pointinghand}] Entrega temario del \emph{Problem set} \#1. Una semana de plazo.


	\item {\color{ForestGreen}{\bf Introducci\'on a modelos lineales en \texttt{R}}}


			\begin{itemize} 
				\item[$\bullet$] {\bf Clase \#8}
					\begin{itemize} 
						\item[$\circ$] Introducci\'on a modelos lineales: \emph{Qu\'e es OLS?}
						\item[$\circ$] {\bf Lecturas}: 
							\begin{itemize}
								\item[$\diamond$] \textcite{Wooldridge2010}: 2.1---2.2.
								% fox ch 5
							\end{itemize}
					\end{itemize}
			\end{itemize}



			%\begin{itemize} 
			%	\item[$\bullet$] {\bf }
			%		\begin{itemize} 
			%			\item[$\circ$] La mec\'anica detr\'as del OLS (I): matrices ``a mano''.
			%			\item[$\circ$] {\bf Lecturas}: 
			%				\begin{itemize}
			%					\item[$\diamond$] \textcite{Namboodiri1984}: Caps. 1 y 2.
						% Monogan2015 Ch. 10.3.1
			%				\end{itemize}
			%		\end{itemize}
			%\end{itemize}

			\begin{itemize} 
				\item[$\bullet$] {\bf Clase \#9}
					\begin{itemize} 
						\item[$\circ$] La mec\'anica detr\'as del OLS (II): matrices en \texttt{R}.
            \item[$\circ$] {\bf Lecturas}: 
            \begin{itemize}
              \item[$\diamond$] \textcite{Namboodiri1984}: Caps. 1 y 2.
            \end{itemize}
% Monogan2015 Ch. 10.3.2
					\end{itemize}
			\end{itemize}



			\begin{itemize} 
				\item[$\bullet$] {\bf Clase \#10}
					\begin{itemize} 
						\item[$\circ$] Coeficientes. % ex's C2.1 Wooldridge.
						\item[$\circ$] {\bf Lecturas}: 
							\begin{itemize}
								\item[$\diamond$] \textcite{Wooldridge2010}: Caps. 3.1---3.2.
								% como interpretar los coeficientes, fox & weisberg 4.3 till p 177
							\end{itemize}
					\end{itemize}
			\end{itemize}



			\begin{itemize} 
				\item[$\bullet$] {\bf Clase \#11}
					\begin{itemize} 
						\item[$\circ$] Error, residual y $\epsilon_{i}$.
					\end{itemize}
			\end{itemize}



			\begin{itemize} 
				\item[$\bullet$] {\bf Clase \#12}
					\begin{itemize} 
						\item[$\circ$] Intervalos de confianza, error est\'andar y \emph{variance-covariance matrix}. % sy_1 W8
						\item[$\circ$] {\bf Lecturas}: 
							\begin{itemize}
								\item[$\diamond$] \textcite{Wooldridge2010}: Cap. 4.3.
								% Fox & weisberg 4.3.1, fox 6.1.3, montgomery 2.4
							\end{itemize}
					\end{itemize}
			\end{itemize}



			\begin{itemize} 
				\item[$\bullet$] {\bf Clase \#13}
					\begin{itemize} 
						\item[$\circ$] Test de hip\'otesis (\emph{t test}), errores Tipo I y II,  significancia estad\'istica (\emph{p-values}). 
						\item[$\circ$] {\bf Lecturas}: 
							\begin{itemize}
								\item[$\diamond$] \textcite{Wooldridge2010}: Cap. 4.2.
							\end{itemize}
					\end{itemize}
			\end{itemize}



			\begin{itemize} 
				\item[$\bullet$] {\bf Clase \#14}
					\begin{itemize} 
						\item[$\circ$] T\'erminos de interacci\'on. Motivaci\'on. Estimaci\'on. Interpretaci\'on.  
						\item[$\circ$] {\bf Lecturas}: 
							\begin{itemize}
								\item[$\diamond$] \textcite{Wooldridge2010}: Cap. 7.4.
								\item[$\diamond$] Thomas Brambor, William Clark and Matt Golder (2006). \href{https://github.com/hbahamonde/OLS/raw/master/Readings/Brambor_et_al.pdf}{\emph{Understanding Interaction Models: Improving Empirical Analyses}}. Political Analysis, 14(1): 63---82.\phantom{\textcite{Brambor2006}}
							\end{itemize}
					\end{itemize}
			\end{itemize}

\item[{\color{red}\Pointinghand}] Entrega temario del \emph{Problem set} \#2. Una semana de plazo.


			\begin{itemize} 
				\item[$\bullet$] {\bf Clase \#15}
					\begin{itemize} 
						\item[$\circ$] Propiedades num\'ericas del OLS, Gauss-Markov, sesgo de variable omitida. % sy_1 W10
						\item[$\circ$] {\bf Lecturas}: 
							\begin{itemize} 
								\item[$\diamond$] \textcite{Wooldridge2010}: pp. 89---94, 102---104.
							\end{itemize}
					\end{itemize}
			\end{itemize}



			\begin{itemize} 
				\item[$\bullet$] {\bf Clase \#16}
					\begin{itemize} 
						\item[$\circ$] \emph{Goodness of fit}, ``coeficiente de determinaci\'on'' (r$^2$), predicci\'on. 
						\item[$\circ$] {\bf Lecturas}:
							\begin{itemize} 
								\item[$\diamond$] \textcite{Wooldridge2010}: pp. 40---41, Cap. 6.3.
								\item[$\diamond$] Gary King (1986). \href{https://github.com/hbahamonde/OLS/raw/master/Readings/King.pdf}{\emph{How Not to Lie With Statistics: Avoiding Common Mistakes in Quantitative Political Science}}. American Journal of Political Science, 30(3): 666---687.\phantom{\textcite{King1986}}

							\end{itemize}
					\end{itemize}
			\end{itemize}






			\begin{itemize} 
				\item[$\bullet$] {\bf Clase \#17}
					\begin{itemize} 
						\item[$\circ$] Problemas y \emph{post-estimation}: multicolinealidad perfecta, heteroskedasticidad, no linearidad, \emph{outliers}, no normalidad de residuos, auto-correlaci\'on. % sy_1 W14, W15, explicar que multicol causa VARIANCE INFLATION (matrix form),
						\item[$\circ$] {\bf Lecturas}: 
						\begin{itemize}
						\item[$\diamond$] \textcite{Wooldridge2010}: Caps. 8 y 9.5.
						\end{itemize}
					\end{itemize}
			\end{itemize}



      % \begin{itemize} 
      %  \item[$\bullet$] {\bf Clase}
      %    \begin{itemize} 
      %      \item[$\circ$] Presentaci\'on de resultados: tablas, gr\'aficos. Variables independientes categoricas.
      %      \item[$\circ$] {\bf Lecturas}: 
      %        \begin{itemize}
      %          \item[$\diamond$] Jeffrey Wooldridge, 2010. \href{https://github.com/hbahamonde/OLS/raw/master/Readings/Wooldridge.pdf}{\emph{Introducci\'on a la Econometr\'ia. Un Enfoque Moderno}}. Cengage Learning: Cap. 4.6.\phantom{\textcite{Wooldridge2010}}
      %        \end{itemize}
      %  {\color{orange}\item[$\bigstar$] Entrega del temario para la tarea pr\'actica ``grande'': \#3}.
      %    \end{itemize}
      % \end{itemize}

\item[{\color{red}\Pointinghand}] Entrega temario del \emph{Problem set} \#3. Una semana de plazo.


  \item {\color{ForestGreen}{\bf Inferencia causal en \texttt{R}}}


      \begin{itemize} 
        \item[$\bullet$] {\bf Clase \#18}
          \begin{itemize} 
            \item[$\circ$] Inferencia Causal: El \emph{Problema Fundamental} en Inferencia Causal, el Supuesto de la ``Ignorabilidad'' y el ``\emph{Potential Outcomes Framework}''.
            \item[$\circ$] {\bf Lecturas}: 
              \begin{itemize}
                \item[$\diamond$] \textcite{Imbens2015}: Ch. 1.
              \end{itemize}
          \end{itemize}
      \end{itemize}





      \begin{itemize} 
        \item[$\bullet$] {\bf Clase \#19}
          \begin{itemize} 
            \item[$\circ$] Variables instrumentales y \emph{two-stage least squares}.
            \item[$\circ$] {\bf Lecturas}: 
              \begin{itemize}
                \item[$\diamond$] \textcite{Angrist2009}: 4.1---4.2.
              \end{itemize}
          \end{itemize}
      \end{itemize}


\item[{\color{red}\Pointinghand}] Entrega temario del \emph{Problem set} \#4. Una semana de plazo.


      \begin{itemize} 
        \item[$\bullet$] {\bf Clase \#20}
          \begin{itemize} 
            \item[$\circ$] Regression discontinuity designs: \emph{Sharp Designs}.
            \item[$\circ$] {\bf Lecturas}: 
              \begin{itemize}
                \item[$\diamond$] \textcite{Angrist2009}: 6---6.1.
              \end{itemize}
          \end{itemize}
      \end{itemize}


      \begin{itemize} 
        \item[$\bullet$] {\bf Clase \#21}
          \begin{itemize} 
            \item[$\circ$] Regression discontinuity designs: \emph{Fuzzy Designs}.
            \item[$\circ$] {\bf Lecturas}: 
              \begin{itemize}
                \item[$\diamond$] \textcite{Angrist2009}: 6.2.
              \end{itemize}
          \end{itemize}
      \end{itemize}


      \begin{itemize} 
        \item[$\bullet$] {\bf Clase \#22}
          \begin{itemize} 
            \item[$\circ$] Incorporando el elemento \emph{tiempo}: fixed effects, differences-in-differences.
            \item[$\circ$] {\bf Lecturas}: 
              \begin{itemize}
                \item[$\diamond$] \textcite{Angrist2009}: Ch. 5.
              \end{itemize}
          \end{itemize}
      \end{itemize}

\item[{\color{red}\Pointinghand}] Entrega temario del trabajo final.

      \begin{itemize} 
        \item[$\bullet$] {\bf \'Ultima Clase}
          \begin{itemize} 
            \item[$\circ$] Presentaciones Grupales. Formato ``conferencia online''.
        \end{itemize}
      \end{itemize}


			

\end{enumerate}


\newpage
\pagenumbering{roman}
\setcounter{page}{1}
\printbibliography



\end{document}


