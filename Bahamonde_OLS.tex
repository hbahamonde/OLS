%LaTeX Curriculum Vitae Template
%
% Copyright (C) 2004-2009 Jason Blevins <jrblevin@sdf.lonestar.org>
% http://jblevins.org/projects/cv-template/
%
% You may use use this document as a template to create your own CV
% and you may redistribute the source code freely. No attribution is
% required in any resulting documents. I do ask that you please leave
% this notice and the above URL in the source code if you choose to
% redistribute this file.

\documentclass[letterpaper]{article}

\usepackage{hyperref}
\hypersetup{
    bookmarks=true,         % show bookmarks bar?
    unicode=false,          % non-Latin characters in Acrobat’s bookmarks
    pdftoolbar=true,        % show Acrobat’s toolbar?
    pdfmenubar=true,        % show Acrobat’s menu?
    pdffitwindow=true,     % window fit to page when opened
    pdfstartview={FitH},    % fits the width of the page to the window
    pdftitle={My title},    % title
    pdfauthor={Author},     % author
    pdfsubject={Subject},   % subject of the document
    pdfcreator={Creator},   % creator of the document
    pdfproducer={Producer}, % producer of the document
    pdfkeywords={keyword1} {key2} {key3}, % list of keywords
    pdfnewwindow=true,      % links in new window
    colorlinks=true,       % false: boxed links; true: colored links
    linkcolor=blue,          % color of internal links (change box color with linkbordercolor)
    citecolor=blue,        % color of links to bibliography
    filecolor=blue,      % color of file links
    urlcolor=blue           % color of external links
}



\usepackage{geometry}
\usepackage{import} % To import email.
\usepackage{marvosym} % face package
%\usepackage{xcolor,color}
\usepackage{fontawesome}
\usepackage{amssymb} % for bigstar
\usepackage{epigraph}
\usepackage[svgnames]{xcolor}

% Comment the following lines to use the default Computer Modern font
% instead of the Palatino font provided by the mathpazo package.
% Remove the 'osf' bit if you don't like the old style figures.
\usepackage[T1]{fontenc}
\usepackage[sc,osf]{mathpazo}

% Set your name here
\def\name{M\'etodos de Investigaci\'on - AP2107}

% Replace this with a link to your CV if you like, or set it empty
% (as in \def\footerlink{}) to remove the link in the footer:
\def\footerlink{}
% \href{http://www.hectorbahamonde.com}{www.HectorBahamonde.com}

% The following metadata will show up in the PDF properties
\hypersetup{
  colorlinks = true,
  urlcolor = blue,
  pdfauthor = {\name},
  pdfkeywords = {intro to social sciences},
  pdftitle = {\name: Syllabus},
  pdfsubject = {Syllabus},
  pdfpagemode = UseNone
}

\geometry{
  body={6.5in, 8.5in},
  left=1.0in,
  top=1.25in
}

% Customize page headers
\pagestyle{myheadings}
\markright{{\tiny \name}}
\thispagestyle{empty}

% Custom section fonts
\usepackage{sectsty}
\sectionfont{\rmfamily\mdseries\Large}
\subsectionfont{\rmfamily\mdseries\itshape\large}

% Don't indent paragraphs.
\setlength\parindent{0em}

% Make lists without bullets
\renewenvironment{itemize}{
  \begin{list}{}{
    \setlength{\leftmargin}{1.5em}
  }
}{
  \end{list}
}


% email input begin
\newread\fid
\newcommand{\readfile}[1]% #1 = filename
{\bgroup
  \endlinechar=-1
  \openin\fid=#1
  \read\fid to\filetext
  \loop\ifx\empty\filetext\relax% skip over comments
    \read\fid to\filetext
  \repeat
  \closein\fid
  \global\let\filetext=\filetext
\egroup}
\readfile{/Users/hectorbahamonde/Bibliografia_PoliSci/email.txt}
% email input end


%%% bib begin
\usepackage[american]{babel}
\usepackage{csquotes}
%\usepackage[style=chicago-authordate,doi=false,isbn=false,url=false,eprint=false]{biblatex}

\usepackage[authordate,isbn=false,doi=false,url=false,eprint=false]{biblatex-chicago}
\DeclareFieldFormat[article]{title}{\mkbibquote{#1}} % make article titles in quotes
\DeclareFieldFormat[thesis]{title}{\mkbibemph{#1}} % make theses italics

\AtEveryBibitem{\clearfield{month}}
\AtEveryCitekey{\clearfield{month}}

\addbibresource{/Users/hectorbahamonde/Bibliografia_PoliSci/library.bib} 
\addbibresource{/Users/hectorbahamonde/Bibliografia_PoliSci/Bahamonde_BibTex2013.bib} 

% USAGES
%% use \textcite to cite normal
%% \parencite to cite in parentheses
%% \footcite to cite in footnote
%% the default can be modified in autocite=FOO, footnote, for ex. 
%%% bib end




\begin{document}

% Place name at left
%{\huge \name}

% Alternatively, print name centered and bold:
\centerline{\huge \bf \name}

\epigraph{\emph{Statistics: ``science dealing with data about the condition of a state or community''}}{Gottfried Aschenwall, 1770}


\vspace{0.25in}

\begin{minipage}{0.45\linewidth}
 Universidad de O$'$Higgins \\
  Instituto de Ciencias Sociales \\
  Rancagua, Chile\\
  \\
  \\

\end{minipage}
\hspace{4cm}\begin{minipage}{0.45\linewidth}
  \begin{tabular}{ll}
{\bf \'Ultima actualizaci\'on}: \today. \\
 {\bf Descarga la \'ultima versi\'on} \href{https://github.com/hbahamonde/OLS/raw/master/Bahamonde_OLS.pdf}{aqu\'i}.%\\
   %{\bf {\color{red}{\scriptsize Not intended as a definitive version}}} %\\
    \\
    \\
    \\
    \\
    \\
  \end{tabular}
\end{minipage}



\subsection*{Aspectos Log\'isticos}


\vspace{1mm}
{\bf Profesor}: H\'ector Bahamonde, PhD.\\
\texttt{e:}\href{mailto:hector.bahamonde@uoh.cl}{\texttt{hector.bahamonde@uoh.cl}}\\
\texttt{w:}\href{http://www.hectorbahamonde.com}{\texttt{www.HectorBahamonde.com}}\\
\texttt{Zoom ID:} \href{https://us02web.zoom.us/j/9513261038?pwd=S3BSWXQxZW11NC9CRjRoMmd0TkpEZz09}{\texttt{951-326-1038}}.\\
{\bf Office Hours (Zoom)}: Toma una hora \href{https://calendly.com/bahamonde/officehours}{\texttt{aqu\'i}}.


\vspace{5mm}
{\bf Hora de c\'atedra}: Martes 12:00---13:30, Jueves 12:00---13:30\\
{\bf Lugar de c\'atedra}: Zoom (no hay clases presenciales este semestre).\\

{\bf Acceso a materiales del curso}: \href{https://ucampus.uoh.cl/uoh/2020/2/AP2107/1/}{\texttt{aqu\'i}}.

\vspace{5mm}
{\bf Ayudante de c\'atedra (TA)}: Gonzalo Barr\'ia (Mg.).\\
\texttt{e:}\href{mailto:gonzalo.barria@uoh.cl}{\texttt{gonzalo.barria@uoh.cl}}\\
\texttt{Zoom ID:} 988-891-7227.\\
{\bf TA Bio}: Gonzalo Barr\'ia es Cientista Pol\'itico (PUC) y Mag\'ister en Ciencia Pol\'itica (PUC).\\
{\bf Hora de ayudant\'ia}: \emph{On-demand}.\\
{\bf Lugar de ayudant\'ia}: Zoom (no hay ayudant\'ias presenciales este semestre).\\


\vspace{5mm}
{\bf Carrera}:  Administraci\'on P\'ublica.\\
%{\bf Eje de Formaci\'on}: {\color{red}PENDIENTE}.\\
{\bf Semestre/A\~no}: Sexto Semestre/2020.\\
%{\bf Pre-requisitos}: {\color{red}PENDIENTE}.\\
{\bf SCT}: 6.\\
{\bf Horas semanales}: C\'atedra (45-60 minutos v\'ia Zoom), Ayudant\'ia  (45-60 minutos v\'ia Zoom). \\
%{\bf Semanas}:  12.




\subsection*{Motivaci\'on: ¿Por qu\'e tomar este curso?}

\emph{¿Qu\'e efecto tiene la educaci\'on sobre los ingresos? ¿C\'omo podemos evaluar los efectos de una reforma educacional? ¿La legalizaci\'on de las drogas aumenta su consumo? ¿Qu\'e candidato/a ganar\'ia la elecci\'on presidencial si \'esta fuera ma\~nana?} 
\\
\\
Las entidades p\'ublicas gu\'ian sus decisiones estrat\'egicas en base a informaci\'on cuantificable, i.e. datos. Esto ha tomado incluso m\'as importancia en la actualidad, donde ha habido una digitalizaci\'on de los datos sociales. Es fundamental que los cientistas sociales en general sepan c\'omo usar estos datos. A\'un m\'as, el quehacer social en general, est\'a constantemente produciendo datos. Cada vez que usas \emph{Twitter}, pides un \emph{Uber}, env\'ias un e-mail, votas, respondes una encuesta, est\'as produciendo datos sociales. Piensa en lo siguiente: si bien es cierto que hace unos diez a\~nos atr\'as \emph{faltaban} datos, hoy en d\'ia los datos \emph{sobran}. El desaf\'io actual consiste en saber c\'omo analizarlos correctamente, y as\'i ayudar a los tomadores de decisiones. Esto es importante. Ma\~nana tu podr\'ias ser un/a analista en una de las decenas de Departamentos de Estudios repartidas en la administraci\'on del Estado. {\bf Este curso te prepara para ese mundo} (incluyendo el mundo de la consultor\'ia).
\\
\\
Aunque lo que aprenderemos es altamente num\'erico y matem\'atico, no te confundas. Estos m\'etodos no son infalibles, y no nos contar\'an ``la verdad'' (si es que algo as\'i existiera). A\'un necesitas ser muy critico(a). Como ver\'as, {\bf la \emph{estad\'istica inferencial} (que es el objeto de este curso) es un \emph{arte}, no una \emph{ciencia}}. Los n\'umeros nos sugerir\'an ciertas ideas, pero aun as\'i nuestro trabajo ser\'a \emph{interpretar} estos resultados. No seas obediente. Se cr\'itico/a y auto-cr\'itico/a. Sospecha de tus propios resultados y el de los dem\'as. Mal que mal, estaremos haciendo {\bf inferencias} (no \emph{certezas}) estad\'isticas. Como veremos, el fantasma de este semestre se llamar\'a \emph{incertidumbre}. 
\\
\\
Este curso considera un \'enfasis especial en la \emph{causalidad}. La \emph{inferencia causal} ha llegado para quedarse en las ciencias sociales. \emph{¿Bajo qu\'e condiciones podemos decir que X \underline{causa} Y?} M\'as que una cuesti\'on matem\'atica, la causalidad toca en muchos aspectos la filosof\'ia de las ciencias. Este semestre aprenderemos qu\'e relaci\'on tiene la experimentaci\'on con la causalidad, c\'omo podemos hacer experimentos en ciencias sociales, y c\'omo podemos emular un experimento (usando ciertos m\'etodos estad\'isticos) cuando no podemos ni debemos hacer uno.
\\
\\
Honestamente, espero que este curso cautive tu atenci\'on, y simiente tu curiosidad intelectual, sobre todo, mostr\'andote que nuestro objeto de estudio (la sociedad) es apasionante. 
\\
\\
\emph{Bienvenid$@$s!}


\subsection*{Prop\'osito Formativo}

El objetivo de este curso es introducir al/la alumno/a a los m\'etodos econom\'etricos b\'asicos para el an\'alisis de datos. El curso avanza progresivamente en distintos t\'opicos en regresi\'on lineal y m\'etodos no lineales. La principal caracter\'istica es la introducci\'on a modelos de regresi\'on lineal para que en cursos m\'as avanzados puedas estudiar otro tipo de estimaciones.


\subsection*{Objetivos Generales del Curso}

El gran objetivo de este curso, es poder generar en la/el estudiante la capacidad de razonamiento cr\'itico, desde un punto de vista emp\'irico. 
\\
\\
El lenguaje que aprenderemos este semestre ser\'a \texttt{R}, el lenguaje de programaci\'on m\'as usado en las ciencias sociales. Esto tiene varias ventajas. \texttt{R} es gratis y corre en todas las plataformas disponibles. Segundo, es un lenguaje orientado a ``objetos''. Esto significa---tercero---que fuerza al/la estudiante a realmente pensar en el proceso matem\'atico/estad\'istico detr\'as del an\'alisis que se est\'a haciendo. Al contrario de otros \emph{softwares} estad\'isticos como \texttt{SPSS} y \texttt{Stata}, donde el/la usuario(a) simplemente aprieta botones sin saber lo que ocurre realmente, \texttt{R} necesita que le digamos exactamente qu\'e hacer. Y eso es lo que aprenderemos este semestre. Cuarto, si sabes \texttt{R}, te ser\'a absolutamente f\'acil aprender \texttt{Stata} (o \texttt{SPSS}).
\\
\\
%A pesar de estas ventajas, los economistas siguen ocupando \texttt{Stata}. Es por esto que tambi\'en cubriremos las funciones b\'asicas en \texttt{Stata}. Dado que \texttt{Stata} es pagado y que est\'a disponible solamente en nuestro laboratorio de computaci\'on UOH, se contempla la posibilidad de hacer un taller corto, voluntario y sin nota el segundo semestre 2020. 
%\\
%\\
Este curso est\'a dividido en cuatro grandes unidades.


\begin{enumerate}
	\item Funciones b\'asicas en \texttt{R}.
	\item Estad\'istica descriptiva en \texttt{R}.
	\item Introducci\'on a modelos lineales en \texttt{R}.
 	\item Inferencia causal en \texttt{R}.
\end{enumerate}
 

\subsection*{Instalaci\'on de \texttt{R}}

Primero, instala \texttt{R} desde el \href{https://www.r-project.org/}{sitio Web} oficial. Click en ``CRAN'' (extremo superior izquierdo). Selecciona cualquier \emph{mirror}. Por ejemplo, b\'ajalo desde el \emph{0-Cloud}. Despu\'es, baja la interfaz m\'as utilizada, llamada R-Studio. Para esto, anda al \href{https://www.rstudio.com}{sitio Web} oficial, despu\'es \emph{Download R-Studio}, \emph{FREE}, selecciona la versi\'on que sea compatible con tu sistema operativo (Windows, Mac, Ubuntu).


	%\begin{itemize}
	%	\item[{\color{red}\Pointinghand}] Si tu \emph{laptop} no puede cargar \texttt{R}, nuestros laboratorios de computaci\'on UOH disponen del \emph{software}. No tener el software (o un computador para cargarlo) no ser\'an excusa para no presentar tus trabajos. Si debes ocupar el laboratorio, planea tu trabajo de manera eficiente. Por cada trabajo que no entregues, tendr\'as un 1.
	%\end{itemize}


\subsection*{Objetivos Espec\'ificos del Curso}

\begin{enumerate}
  \item Lograr establecer una pregunta pol\'itica/social y un m\'etodo de identificaci\'on que permita verificar la hip\'otesis de forma causal.
  \item Poder \emph{testear} hip\'otesis y tener las herramientas para analizar pol\'iticas de forma cr\'itica.
  \item Entender las limitaciones de los trabajos emp\'iricos y los \emph{trade offs} existentes al establecer supuestos.
\end{enumerate}


\begin{itemize}
    \item[{\color{red}\Pointinghand}] Se espera que los estudiantes hagan sus respectivas lecturas \emph{antes} de cada clase para poder participar en el debate cr\'itico que haremos en cada una de ellas. Tambi\'en se espera que los/las estudiantes hagan los ejercicios pr\'acticos clase a clase.
\end{itemize}





\subsection*{Integridad Acad\'emica}


\begin{itemize}
	\item[$\circ$] El plagio y la copia ser\'an sancionadas con un 1. En caso de duda pregunta a tu profesor/ayudante. Procura citar todo lo que no sea de tu propiedad intelectual.
	\item[$\circ$] No se aceptan trabajos atrasados. Si tienes problemas de conectividad, planifica tus env\'ios con anticipaci\'on. S\'olo se revisar\'a lo que est\'e subido a uCampus (aunque est\'e incompleto). Si no hay nada, tendr\'as un 1.
	\item[$\circ$] Ni el ayudante ni el profesor est\'an obligados a responder preguntas (a) despu\'es de las 5 pm durante d\'ias de semana, (b) durante fines de semana, (c) festivos.
\end{itemize}

\begin{itemize}
\item[{\color{red}\Pointinghand}] No existir\'an excepciones. Planifica tu trabajo responsablemente. 
\end{itemize}

\subsection*{Pol\'itica sobre Trabajo Cooperativo}

{\bf Yo recomiendo el trabajo cooperativo}. Es saludable que consultes con tus compa\~neros/as de curso, y que traten, en la medida de lo posible, de encontrar las soluciones en conjunto. Sin embargo, salvo por el examen final y la presentaci\'on final (m\'as sobre esto abajo), todos los trabajos (y sus evaluaciones) ser\'an individuales.


\subsection*{Ayudant\'ia}

Las ayudant\'ias se har\'an por \emph{Zoom}. Y se har\'an a pedido de los ayudantes. Pero en general, espera tener al menos dos ayudant\'ias al mes. En ellas se ver\'an aspectos relacionados al curso.



\subsection*{Evaluaciones}

\begin{enumerate}

	% Participation
	\item {\bf Lecturas y Participaci\'on }: 10\%.
	
    El TA y yo asumiremos durante todo el semestre que has le\'ido. Nosotros empleamos un m\'etodo de clases interactivo, pero este m\'etodo necesita de tu participaci\'on activa en clases.
    \\
    \\  
    Si no puedes asistir a la clase sincr\'onica, existir\'an opciones para dejar entradas en la secci\'on \emph{Foro} de \texttt{uCampus}.

	\item {\bf \emph{Problem Sets}}: 10\% cada uno, 40\% en total.

Estos \emph{problem sets} son ejercicios pr\'acticos. Nosotros te entregaremos un \emph{script} de \texttt{R} junto a una base de datos. T\'u tendr\'as que resolver las preguntas dentro de \texttt{R} y devolvernos ese \emph{script}. El ayudante y el profesor estaran disponibles para resolver preguntas v\'ia email o Zoom.

\begin{itemize}
		\item[$\diamond$] {\bf Aunque no es necesario, s\'i puedes ocupar recursos externos, como Internet}.
		\item[$\diamond$] Es importante que estas l\'ineas corran bien: el usuario (yo) tiene que ser cap\'az de ver c\'omo \texttt{R} ejecuta cada linea, sin estancarse.
		\item[$\diamond$] Es importante que vayas guiando al usuario (yo) sobre tu raciocinio. Aseg\'urate de comentar (usando el simbolo \#).
\end{itemize}


\item {\bf Un trabajo final obligatorio/no-eximible (30\%) y una presentaci\'on final (20\%, v\'ia Zoom)}: 50\% en total.\\


En este curso, la actividad final es un trabajo final (30\%) que tiene formato de trabajo grupal. Usando una base de datos que nosotros te daremos, t\'u y tu grupo deber\'an responder una serie de preguntas. El producto final (i.e. lo que debes entregar) consiste en un \emph{script} de \texttt{R}. La nota es grupal (i.e. todo el grupo recibir\'a la misma nota). {\bf Los grupos ser\'an de 2 personas}. La formacion del grupo es end\'ogena.
\\
\\
El paper (\emph{script}) se puede entregar antes, pero una vez cerrado el plazo, no se recibir\'an trabajos. Los \emph{scripts} que se entreguen tarde o v\'ia \emph{email} tendr\'an un 1 (sin opci\'on a reclamo). {\bf No hay excepciones}. 
\\
\\
En un formato muy parecido a una conferencia acad\'emica (virtual, no presencial), tendr\'as (junto a tu grupo) que presentar los principales hallazgos (20\%). Todos/as presentan. Cada presentaci\'on debe durar no menos de 15 minutos, pero nunca m\'as de 20 minutos. Las presentaciones se realizar\'an virtualmente (i.e. v\'ia Zoom) el \'ultimo d\'ia de clases. Tendr\'as que ocupar \emph{slides} (``Power Point''). Para tales efectos, tendr\'as que compartir pantalla desde tu casa, y hacer tu presentaci\'on de esa manera.



%\\
%\\
Les recomiendo ``verme'' (v\'ia Zoom) en \href{https://calendly.com/bahamonde/officehours}{mis office hours} \emph{antes} del plazo de entrega. Si quieres, \href{mailto:\filetext}{env\'iame un email} con tu borrador, y yo te devolver\'e comentarios. V\'elo como una pre-correcci\'on. Esto es voluntario. Tambi\'en puedes contactar al/la TA. {\bf No se procesar\'an preguntas durante fines de semana, y/o festivos}.


\end{enumerate}


\underline{En resumen}:

\begin{table}[h]
\centering
\begin{tabular}{ccc}
							& \textbf{Porcentaje} & {\bf Porcentaje Acumulado} \\
							\hline
Participaci\'on (c\'atedra, foro \texttt{uCampus} y ayudant\'ia) 	 & 10\%       	 & 10\% \\
\hline
\emph{Problem Set} \#1 													 & 10\% 		 & 20\%  \\
\emph{Problem Set} \#2 													 & 10\% 		 & 30\%  \\
\emph{Problem Set} \#3 													 & 10\% 		 & 40\%  \\
\emph{Problem Set} \#4 													 & 10\% 		 & 50\%  \\
\hline
Trabajo final grupal 												 & 30\% 	 	 & 80\% \\
Presentaci\'on grupal												 & 20\% 	 	 & 100\% \\
\hline             
\end{tabular}
\end{table}


\subsection*{Textos M\'inimos}

\begin{itemize}
  \item[$\bullet$] Guido Imbens and Donald Rubin (1998). \href{https://github.com/hbahamonde/OLS/raw/master/Readings/Imbens_Rubin.pdf}{\emph{Causal Inference for Statistics, Social, and Biomedical Sciences}}.\phantom{\textcite{Imbens2015}}
  \item[$\bullet$] Joshua Angrist and Jorn-Steffen Pischke (2009). \href{https://github.com/hbahamonde/OLS/raw/master/Readings/MHE.pdf}{\emph{Mostly Harmless Econometrics: An Empiricist's Companion}}.\phantom{\textcite{Angrist2009}}
  \item[$\bullet$] Jeffrey Wooldridge (2010). \href{https://github.com/hbahamonde/OLS/raw/master/Readings/Wooldridge.pdf}{\emph{Introducci\'on a la Econometr\'ia. Un Enfoque Moderno}}.\phantom{\textcite{Wooldridge2010}}
  \item[$\bullet$] Urdinez y Cruz (2019). \href{https://arcruz0.github.io/libroadp/index.html}{\emph{AnalizaR Datos Pol\'iticos}}.\phantom{\textcite{Urdinez:2019aa}}
  \item[$\bullet$] Krishnan Namboodiri (1984). \href{https://github.com/hbahamonde/OLS/raw/master/Readings/Namboodiri.pdf}{\emph{Matrix Algebra, an Introduction}}.\phantom{\textcite{Namboodiri1984}}

\end{itemize}

\subsection*{Textos Recomendados}

\begin{itemize}
  \item[$\bullet$] Paul Rosenbaum (2010). \href{https://github.com/hbahamonde/OLS/raw/master/Readings/Rosenbaum.pdf}{\emph{Design of Observational Studies}}.\phantom{\textcite{Rosenbaum2010a}}
  \item[$\bullet$] James Monogan (2015). \href{https://github.com/hbahamonde/OLS/raw/master/Readings/Monogan.pdf}{\emph{Political Analysis Using R}}.\phantom{\textcite{Monogan2015}}
\end{itemize}


\begin{itemize}
\item[{\color{red}\Pointinghand}] Tambi\'en se considerar\'an algunos \emph{papers}. Estos estar\'an se\~nalados en las fechas indicadas y en la secci\'on de Bibliograf\'ia.
\end{itemize}


\subsection*{Calendario}


\begin{enumerate}
	\item {\color{ForestGreen}{\bf Funciones b\'asicas en \texttt{R}}}

			\begin{itemize} 
				\item[$\bullet$] {\bf Clase \#1}
				\begin{itemize} 
					\item[$\circ$] Introducciones: programa de curso, requerimientos, expectativas, etc.
					\item[$\circ$] \emph{Qu\'e es \texttt{R}?} Instalaci\'on de \texttt{R} y \texttt{RStudio}.
          \item[$\circ$] \emph{Qu\'e es \texttt{Stata}?}
					\item[$\circ$] {\bf Lecturas}: 
						\begin{itemize} 
							\item[$\diamond$] \textcite{Wooldridge2010}: Cap. 1.
							\item[$\diamond$] \textcite{Urdinez:2019aa}: Cap. 2.
						\end{itemize}
					% Monogan2015 Ch. 1
				\end{itemize}
			\end{itemize}






			\begin{itemize} 
				\item[$\bullet$] {\bf Clase \#2}
				\begin{itemize} 
					\item[$\circ$] Funciones b\'asicas: promedio, \texttt{help()}, operadores, tipos de objetos (\emph{character}, \emph{arrays}, fechas, listas, \emph{dataframes}).
					\item[$\circ$] Cargando bases de datos (I): formatos, etiquetas, tipos de variables, descripci\'on b\'asica. % ch 2 fox r companion, Monogan2015 Ch 2, https://stats.idre.ucla.edu/stat/data/intro_r/intro_r.html#(12)
					\item[$\circ$] {\bf Lecturas}: 
					\begin{itemize}
						\item[$\diamond$] \textcite{Urdinez:2019aa}: Cap. 5.
					\end{itemize}
				\end{itemize}
			\end{itemize}




			\begin{itemize} 
				\item[$\bullet$] {\bf Clase \#3}
					\begin{itemize} 
				\item[$\circ$] Cargando bases de datos (II): transformaciones, creaci\'on de nuevas variables.
				\item[$\circ$] Manipulando bases de datos: generaci\'on de matrices y \emph{dataframes}, \texttt{merge}, \texttt{append}. Logs.  %(p. 35 Gill, Essential Math)
					\end{itemize}
					% Monogan2015 Ch 2, 
			\end{itemize}




			\begin{itemize} 
				\item[$\bullet$] {\bf Clase \#4}
					\begin{itemize} 
						\item[$\circ$] Visualizaci\'on de datos (I): \emph{bar plots} (variable categ\'orica/continua, categ\'orica/categ\'orica), \emph{scatter plots}, histogramas, \emph{time series plots}. % fox Applied: ch. 3, fox companion ch's 3.1-3.3, Monogan2015 Ch 3
						\item[$\circ$] {\bf Lecturas}:
							\begin{itemize}
								\item[$\diamond$] \textcite{Urdinez:2019aa}: Cap. 4.
							\end{itemize}
					\end{itemize}
			\end{itemize}



			\begin{itemize} 
				\item[$\bullet$] {\bf Clase \#5}
					\begin{itemize} 
				\item[$\circ$] Visualizaci\'on de datos (II): \emph{plots} m\'as complejos (por categor\'ias), mapas.
				% Monogan2015 Ch 3, fox ch 3
				\item[$\circ$] {\bf Lecturas}: 
					\begin{itemize}
						\item[$\diamond$] \textcite{Urdinez:2019aa}: Cap. 15.
					\end{itemize}
					\end{itemize}
			\end{itemize}


	\item {\color{ForestGreen}{\bf Estad\'istica descriptiva en \texttt{R}}}

			\begin{itemize} 
				\item[$\bullet$] {\bf Clase \#6}
					\begin{itemize} 
				\item[$\circ$] Estad\'istica descriptiva (I): Teor\'ia de probabilidades: distribuciones, varianza. % sy_1/week 5, Monogan2015 Ch 4, Fox_Appendices (App D), Chs. 7  (Gill, Essential Math)
					\end{itemize}
			\end{itemize}


			\begin{itemize} 
				\item[$\bullet$] {\bf Clase \#7}
					\begin{itemize} 
				\item[$\circ$] Estad\'istica descriptiva (II): binomial, normal, otras; simulaci\'on. % sy_1/week 6, fox applie ch. 4, Fox_Appendices (App D)., Chs. 8 (Gill, Essential Math).
					\end{itemize}
			\end{itemize}


\item[{\color{red}\Pointinghand}] Entrega temario del \emph{Problem set} \#1. Una semana de plazo.


	\item {\color{ForestGreen}{\bf Introducci\'on a modelos lineales en \texttt{R}}}


			\begin{itemize} 
				\item[$\bullet$] {\bf Clase \#8}
					\begin{itemize} 
						\item[$\circ$] Introducci\'on a modelos lineales: \emph{Qu\'e es OLS?}
						\item[$\circ$] {\bf Lecturas}: 
							\begin{itemize}
								\item[$\diamond$] \textcite{Wooldridge2010}: 2.1---2.2.
								% fox ch 5
							\end{itemize}
					\end{itemize}
			\end{itemize}



			%\begin{itemize} 
			%	\item[$\bullet$] {\bf }
			%		\begin{itemize} 
			%			\item[$\circ$] La mec\'anica detr\'as del OLS (I): matrices ``a mano''.
			%			\item[$\circ$] {\bf Lecturas}: 
			%				\begin{itemize}
			%					\item[$\diamond$] \textcite{Namboodiri1984}: Caps. 1 y 2.
						% Monogan2015 Ch. 10.3.1
			%				\end{itemize}
			%		\end{itemize}
			%\end{itemize}

			\begin{itemize} 
				\item[$\bullet$] {\bf Clase \#9}
					\begin{itemize} 
						\item[$\circ$] La mec\'anica detr\'as del OLS (II): matrices en \texttt{R}.
            \item[$\circ$] {\bf Lecturas}: 
            \begin{itemize}
              \item[$\diamond$] \textcite{Namboodiri1984}: Caps. 1 y 2.
            \end{itemize}
% Monogan2015 Ch. 10.3.2
					\end{itemize}
			\end{itemize}



			\begin{itemize} 
				\item[$\bullet$] {\bf Clase \#10}
					\begin{itemize} 
						\item[$\circ$] Coeficientes. % ex's C2.1 Wooldridge.
						\item[$\circ$] {\bf Lecturas}: 
							\begin{itemize}
								\item[$\diamond$] \textcite{Wooldridge2010}: Caps. 3.1---3.2.
								% como interpretar los coeficientes, fox & weisberg 4.3 till p 177
							\end{itemize}
					\end{itemize}
			\end{itemize}



			\begin{itemize} 
				\item[$\bullet$] {\bf Clase \#11}
					\begin{itemize} 
						\item[$\circ$] Error, residual y $\epsilon_{i}$.
					\end{itemize}
			\end{itemize}



			\begin{itemize} 
				\item[$\bullet$] {\bf Clase \#12}
					\begin{itemize} 
						\item[$\circ$] Intervalos de confianza. % sy_1 W8
						\item[$\circ$] {\bf Lecturas}: 
							\begin{itemize}
								\item[$\diamond$] \textcite{Wooldridge2010}: Cap. 4.3.
								% Fox & weisberg 4.3.1, fox 6.1.3, montgomery 2.4
							\end{itemize}
					\end{itemize}
			\end{itemize}



			\begin{itemize} 
				\item[$\bullet$] {\bf Clase \#13}
					\begin{itemize} 
						\item[$\circ$] Test de hip\'otesis (\emph{t test}), errores Tipo I y II,  significaci\'on estad\'istica (\emph{p-values}). 
						\item[$\circ$] {\bf Lecturas}: 
							\begin{itemize}
								\item[$\diamond$] \textcite{Wooldridge2010}: Cap. 4.2.
							\end{itemize}
					\end{itemize}
			\end{itemize}



			\begin{itemize} 
				\item[$\bullet$] {\bf Clase \#14}
					\begin{itemize} 
						\item[$\circ$] T\'erminos de interacci\'on. Motivaci\'on. Estimaci\'on. Interpretaci\'on.  
						\item[$\circ$] {\bf Lecturas}: 
							\begin{itemize}
								\item[$\diamond$] \textcite{Wooldridge2010}: Cap. 7.4.
								\item[$\diamond$] Thomas Brambor, William Clark and Matt Golder (2006). \href{https://github.com/hbahamonde/OLS/raw/master/Readings/Brambor_et_al.pdf}{\emph{Understanding Interaction Models: Improving Empirical Analyses}}. Political Analysis, 14(1): 63---82.\phantom{\textcite{Brambor2006}}
							\end{itemize}
					\end{itemize}
			\end{itemize}

\item[{\color{red}\Pointinghand}] Entrega temario del \emph{Problem set} \#2. Una semana de plazo.


			\begin{itemize} 
				\item[$\bullet$] {\bf Clase \#15}
					\begin{itemize} 
						\item[$\circ$] Propiedades num\'ericas del OLS, Gauss-Markov, sesgo de variable omitida. % sy_1 W10
						\item[$\circ$] {\bf Lecturas}: 
							\begin{itemize} 
								\item[$\diamond$] \textcite{Wooldridge2010}: pp. 89---94, 102---104.
							\end{itemize}
					\end{itemize}
			\end{itemize}



			\begin{itemize} 
				\item[$\bullet$] {\bf Clase \#16}
					\begin{itemize} 
						\item[$\circ$] \emph{Goodness of fit}, ``coeficiente de determinaci\'on'' (r$^2$), predicci\'on. 
						\item[$\circ$] {\bf Lecturas}:
							\begin{itemize} 
								\item[$\diamond$] \textcite{Wooldridge2010}: pp. 40---41, Cap. 6.3.
								\item[$\diamond$] Gary King (1986). \href{https://github.com/hbahamonde/OLS/raw/master/Readings/King.pdf}{\emph{How Not to Lie With Statistics: Avoiding Common Mistakes in Quantitative Political Science}}. American Journal of Political Science, 30(3): 666---687.\phantom{\textcite{King1986}}

							\end{itemize}
					\end{itemize}
			\end{itemize}






			\begin{itemize} 
				\item[$\bullet$] {\bf Clase \#17}
					\begin{itemize} 
						\item[$\circ$] Problemas y \emph{post-estimation}: multicolinealidad perfecta, heteroskedasticidad, no linearidad, \emph{outliers}, no normalidad de residuos, auto-correlaci\'on. % sy_1 W14, W15, explicar que multicol causa VARIANCE INFLATION (matrix form),
						\item[$\circ$] {\bf Lecturas}: 
						\begin{itemize}
						\item[$\diamond$] \textcite{Wooldridge2010}: Caps. 8 y 9.5.
						\end{itemize}
					\end{itemize}
			\end{itemize}



      % \begin{itemize} 
      %  \item[$\bullet$] {\bf Clase}
      %    \begin{itemize} 
      %      \item[$\circ$] Presentaci\'on de resultados: tablas, gr\'aficos. Variables independientes categoricas.
      %      \item[$\circ$] {\bf Lecturas}: 
      %        \begin{itemize}
      %          \item[$\diamond$] Jeffrey Wooldridge, 2010. \href{https://github.com/hbahamonde/OLS/raw/master/Readings/Wooldridge.pdf}{\emph{Introducci\'on a la Econometr\'ia. Un Enfoque Moderno}}. Cengage Learning: Cap. 4.6.\phantom{\textcite{Wooldridge2010}}
      %        \end{itemize}
      %  {\color{orange}\item[$\bigstar$] Entrega del temario para la tarea pr\'actica ``grande'': \#3}.
      %    \end{itemize}
      % \end{itemize}

\item[{\color{red}\Pointinghand}] Entrega temario del \emph{Problem set} \#3. Una semana de plazo.


  \item {\color{ForestGreen}{\bf Inferencia causal en \texttt{R}}}


      \begin{itemize} 
        \item[$\bullet$] {\bf Clase \#18}
          \begin{itemize} 
            \item[$\circ$] Inferencia Causal: El \emph{Problema Fundamental} en Inferencia Causal, el Supuesto de la ``Ignorabilidad'' y el ``\emph{Potential Outcomes Framework}''.
            \item[$\circ$] {\bf Lecturas}: 
              \begin{itemize}
                \item[$\diamond$] \textcite{Imbens2015}: Ch. 1.
              \end{itemize}
          \end{itemize}
      \end{itemize}





      \begin{itemize} 
        \item[$\bullet$] {\bf Clase \#19}
          \begin{itemize} 
            \item[$\circ$] Variables instrumentales y \emph{two-stage least squares}.
            \item[$\circ$] {\bf Lecturas}: 
              \begin{itemize}
                \item[$\diamond$] \textcite{Angrist2009}: 4.1---4.2.
              \end{itemize}
          \end{itemize}
      \end{itemize}


\item[{\color{red}\Pointinghand}] Entrega temario del \emph{Problem set} \#4. Una semana de plazo.


      \begin{itemize} 
        \item[$\bullet$] {\bf Clase \#20}
          \begin{itemize} 
            \item[$\circ$] Regression discontinuity designs: \emph{Sharp Designs}.
            \item[$\circ$] {\bf Lecturas}: 
              \begin{itemize}
                \item[$\diamond$] \textcite{Angrist2009}: 6---6.1.
              \end{itemize}
          \end{itemize}
      \end{itemize}


      \begin{itemize} 
        \item[$\bullet$] {\bf Clase \#21}
          \begin{itemize} 
            \item[$\circ$] Regression discontinuity designs: \emph{Fuzzy Designs}.
            \item[$\circ$] {\bf Lecturas}: 
              \begin{itemize}
                \item[$\diamond$] \textcite{Angrist2009}: 6.2.
              \end{itemize}
          \end{itemize}
      \end{itemize}


      \begin{itemize} 
        \item[$\bullet$] {\bf Clase \#22}
          \begin{itemize} 
            \item[$\circ$] Incorporando el elemento \emph{tiempo}: fixed effects, differences-in-differences.
            \item[$\circ$] {\bf Lecturas}: 
              \begin{itemize}
                \item[$\diamond$] \textcite{Angrist2009}: Ch. 5.
              \end{itemize}
          \end{itemize}
      \end{itemize}

\item[{\color{red}\Pointinghand}] Entrega temario del trabajo final.

      \begin{itemize} 
        \item[$\bullet$] {\bf \'Ultima Clase}
          \begin{itemize} 
            \item[$\circ$] Presentaciones Grupales. Formato ``conferencia online''.
        \end{itemize}
      \end{itemize}


			

\end{enumerate}


\newpage
\pagenumbering{roman}
\setcounter{page}{1}
\printbibliography



\end{document}


