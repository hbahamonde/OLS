%----------------------------------------------------------------------------------------
%	PACKAGES AND OTHER DOCUMENT CONFIGURATIONS
%----------------------------------------------------------------------------------------

\documentclass[10pt]{article}
\usepackage{lipsum} % Package to generate dummy text throughout this template

%\usepackage[light, math]{iwona}
%\usepackage[sc]{mathpazo} % Use the Palatino font
\usepackage[T1]{fontenc} % Use 8-bit encoding that has 256 glyphs
\linespread{1.05} % Line spacing - Palatino needs more space between lines
\usepackage{microtype} % Slightly tweak font spacing for aesthetics
\usepackage{multirow}
\usepackage[hmarginratio=1:1,top=32mm,columnsep=20pt]{geometry} % Document margins
\usepackage{multicol} % Used for the two-column layout of the document
\usepackage[hang, small,labelfont=bf,up,textfont=it,up]{caption} % Custom captions under/above floats in tables or figures
\usepackage{booktabs} % Horizontal rules in tables
\usepackage{float} % Required for tables and figures in the multi-column environment - they need to be placed in specific locations with the [H] (e.g. \begin{table}[H])

\usepackage{lettrine} % The lettrine is the first enlarged letter at the beginning of the text
\usepackage{paralist} % Used for the compactitem environment which makes bullet points with less space between them

\usepackage{abstract} % Allows abstract customization
\renewcommand{\abstractnamefont}{\normalfont\bfseries} % Set the "Abstract" text to bold
\renewcommand{\abstracttextfont}{\normalfont\small\itshape} % Set the abstract itself to small italic text

\usepackage{titlesec} % Allows customization of titles
\renewcommand\thesection{\Roman{section}} % Roman numerals for the sections
\renewcommand\thesubsection{\Roman{subsection}} % Roman numerals for subsections
\titleformat{\section}[block]{\large\scshape\centering}{\thesection.}{1em}{} % Change the look of the section titles
\titleformat{\subsection}[block]{\large}{\thesubsection.}{1em}{} % Change the look of the section titles

\usepackage{fancybox, fancyvrb, calc}
\usepackage[svgnames]{xcolor}


%----------------------------------------------------------------------------------------
%	DOCUMENT ID (Department, Professor, Course, etc.) 
%----------------------------------------------------------------------------------------

\usepackage{fancyhdr} % Headers and footers
\pagestyle{fancy} % All pages have headers and footers
\fancyhead{} % Blank out the default header
\fancyfoot{} % Blank out the default footer
\fancyhead[C]{$\bullet$ Final Assignment $\bullet$} % Custom header text
\fancyfoot[RO,LE]{\thepage} % Custom footer text

%----------------------------------------------------------------------------------------
%	MY PACKAGES 
%----------------------------------------------------------------------------------------

\usepackage{amsmath}	
%\usepackage{rotating}
\usepackage{textcomp}
\usepackage{caption}
\usepackage{etex}
%\usepackage[export]{adjustbox}
%\usepackage{afterpage}
%\usepackage{filecontents}
\usepackage{color}
\usepackage{latexsym}
\usepackage{lscape}				%\begin{landscape} and \end{landscape}
\usepackage{amsfonts}
%\usepackage{mathabx}
\usepackage{amssymb}
%\usepackage{dashrule}
%\usepackage{txfonts}
%\usepackage{pgfkeys}
%\usepackage{framed}
\usepackage{tree-dvips}
\usepackage{caption}
%\usepackage{fancyvrb}
%\usepackage{pgffor}
\usepackage{xcolor}
%\usepackage{pxfonts}
\usepackage{wasysym}
\usepackage{authblk}
%\usepackage{paracol}
\usepackage{setspace}
%\usepackage{qtree}
%\usepackage{tree-dvips}
\usepackage{sgame}				% shouldn't have neither array nor tabularx packages
\usepackage{tikz}
%\usetikzlibrary{trees}
\usepackage[latin1]{inputenc}
%\label{tab:1} 		%\autoref{tab:1}	%ocupar para citar.
% \hyperlik{table1}	\hypertarget{table1} 
% \textquoteright			%apostrofe
\usepackage{hyperref} 		%desactivar para link rojos
\usepackage{natbib}
%\usepackage{proof} 			%for proofs





%----------------------------------------------------------------------------------------
%	Other ADDS-ON
%----------------------------------------------------------------------------------------

% independence symbol \independent
\newcommand\independent{\protect\mathpalette{\protect\independenT}{\perp}}
\def\independenT#1#2{\mathrel{\rlap{$#1#2$}\mkern2mu{#1#2}}}


% VERBATIM WITH BACKGROUND COLOR
\newenvironment{colframe}{%
  \begin{Sbox}
    \begin{minipage}
      {\columnwidth%-\leftmargin-\rightmargin-6pt
      }
    }{%
    \end{minipage}
  \end{Sbox}
  \begin{center}
    \colorbox{LightSteelBlue}{\TheSbox}
  \end{center}
}


\hypersetup{
    bookmarks=true,         % show bookmarks bar?
    unicode=false,          % non-Latin characters in Acrobat$'$s bookmarks
    pdftoolbar=true,        % show Acrobat$'$s toolbar?
    pdfmenubar=true,        % show Acrobat$'$s menu?
    pdffitwindow=false,     % window fit to page when opened
    pdfstartview={FitH},    % fits the width of the page to the window
    pdftitle={My title},    % title
    pdfauthor={Author},     % author
    pdfsubject={Subject},   % subject of the document
    pdfcreator={Creator},   % creator of the document
    pdfproducer={Producer}, % producer of the document
    pdfkeywords={keyword1} {key2} {key3}, % list of keywords
    pdfnewwindow=true,      % links in new window
    colorlinks=true,       % false: boxed links; true: colored links
    linkcolor=ForestGreen,          % color of internal links (change box color with linkbordercolor)
    citecolor=ForestGreen,        % color of links to bibliography
    filecolor=ForestGreen,      % color of file links
    urlcolor=ForestGreen           % color of external links
}


% PROPOSITIONS
\newtheorem{proposition}{Proposition}

%\linespread{1.5}

%----------------------------------------------------------------------------------------
%	TITLE SECTION
%----------------------------------------------------------------------------------------

%\title{\vspace{-15mm}\fontsize{18pt}{7pt}\selectfont\textbf{Experimental Economists and Psychologists: Two Worlds Apart}} % Article title

%\author[1]{
%\large
%\textsc{H\'ector Bahamonde}\\ 
%\thanks{}
%\normalsize Political Science Dpt. $\bullet$ Rutgers University \\ % Your institution
%\normalsize \texttt{e:}\href{mailto:hector.bahamonde@rutgers.edu}{\texttt{hector.bahamonde@rutgers.edu}}\\
%\normalsize \texttt{w:}\href{http://www.hectorbahamonde.com}{\texttt{www.hectorbahamonde.com}}
%\vspace{-5mm}
%}
%\date{\today}

%----------------------------------------------------------------------------------------

\begin{document}

%\maketitle % Insert title


\thispagestyle{fancy} % All pages have headers and footers

%----------------------------------------------------------------------------------------
%	ABSTRACT
%----------------------------------------------------------------------------------------

%\begin{abstract}
%	ABSTRACT
%\end{abstract}


%----------------------------------------------------------------------------------------
%	CONTENT
%----------------------------------------------------------------------------------------

%\graphicspath{
%{/Users/hectorbahamonde/RU/Term5/Experiments_Redlawsk/Experiment/Data/}
%}
\hspace{-5mm}{\bf Professor}: H\'ector Bahamonde, PhD.\\
\texttt{e:}\href{mailto:hibano@utu.fi}{\texttt{hibano@utu.fi}}\\
\texttt{w:}\href{http://www.hectorbahamonde.com}{\texttt{www.HectorBahamonde.com}}\\
{\bf Curso}: OLS.\\
\hspace{-5mm}{\bf TA}: Valterri Pulkkinen.


\vspace{-0.8cm}
\section*{Instructions}

Using the randomly assigned dataset, and within the deadline established in the program, you will:

\begin{enumerate}
	\item {\bf Generate a research question}. An example might be \emph{What explains Y?} To answer this question, use the \emph{codebook}, and see which of the included variables might help you answer your research question. Also, remember to articulate your work based on the variables you do have.
	
	\item {\bf Generate a working hypothesis}. A working hypothesis is an answer (an assertion). An example might be \emph{$Y$ can be explained because of $X_{1}$}. In general, here you also explain \emph{why} you think that's the case.
	
	\item  {\bf Generate an alternative hypothesis}. Just like the working hypothesis, the working one is an assertion. The difference is that the alternative answers the question differently. For instance, here you can hypothesize that $Y$ is rather explained by $X_{2}$ (not $X_{1}$).
	
	\item {\bf Hypothesis testing}. To do this you will need to perform at least one OLS model. Remember, $Y$ should be continuous. Also, remember to use control variables. Always justify your control variables (tell us \emph{why} you decided to include them). Maybe a (super) quick literature review and a short analysis of the problem might guide your decision-making process.


	\item {\bf Presentation}.  \emph{After testing both hypothesis, for which one do you find statistical support?} To answer this question, look at the (1) p-values, (2) standard errors, (3) r$^{2}$ of the model, and (4) effect sizes ($\beta$). Do \emph{not} copy and paste \texttt{R} output into your presentation. Present a carefully designed regression table.

	\item {\bf Post-estimation}. Using a scatter plot, check (1) normality of your residuals/errors, and (2) homo/hetero-scedasticity. (a) Answer in your presentation: \emph{Do your residuals behave according to the assumptions behind OLS theory?}

	\item {\bf In-class presentation (``conference'')}. In the presentation you should talk about all these issues. Each presentation should last approximately 10-15 minutes. You will have to convince us that you did everything that's possible to find an appropriate model and that you tested all your assumptions. The rest of the class will not only listen to the presentation, but also will contribute to the discussion via constructive criticism. 

	\item {\bf Products}: 

		\begin{enumerate}
			\item {\bf A script}: submit a \texttt{R} script with all your data analyses. Remember that the script should run properly and should not get stuck. For that, include in your code all the packages and libraries so I can replicate your work.  
			\item {\bf One presentation}: a conference-style presentation with a Power Point.
		\end{enumerate}

\end{enumerate}

\newpage
\section*{Datasets and Dependent Variables}


%\afterpage{
\begin{table}[ph!]
\begin{center}
\begin{scriptsize}
{\renewcommand{\arraystretch}{2}%
\begin{tabular}{  c |  c | c | c }
\toprule
\textbf{Library}             & \textbf{Dataset}  & \textbf{Codebook} & \textbf{Dependent Variable} \\
\midrule
\multirow{6}{*}{\texttt{library(AER)}}  & \texttt{data(Fatalities)}      & \texttt{help(Fatalities)}      & \texttt{Fatalities\$fatal} \\\cline{2-4}
                                        & \texttt{data(Guns)}            & \texttt{help(Guns)}            & \texttt{Guns\$violent} \\\cline{2-4}
                                        & \texttt{data(HousePrices)}     & \texttt{help(HousePrices)}     & \texttt{HousePrices\$price} \\\cline{2-4}
                                        & \texttt{data(Journals)}        & \texttt{help(Journals)}        & \texttt{Journals\$price} \\\cline{2-4}
                                        & \texttt{data(TeachingRatings)} & \texttt{help(TeachingRatings)} & \texttt{TeachingRatings\$beauty} \\\cline{2-4}
\bottomrule
\end{tabular}}
\end{scriptsize}
\end{center}
\end{table}
%}


\end{document}